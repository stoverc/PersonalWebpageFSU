\documentclass[12pt,oneside]{amsart}
\usepackage[margin=0.5in]{geometry}
\usepackage{lmodern}
\usepackage[T1]{fontenc}
\usepackage{amsmath,amssymb,xypic,xspace,graphicx,url,changepage,stoversymb}
\usepackage{nopageno}
\usepackage[inline]{enumitem}
\usepackage{soul}
\parskip=12pt
%\setlength{\textwidth}{7.25in}
%\setlength{\textheight}{9.25in}
%\setlength{\topmargin}{-0.75in}
%\setlength{\oddsidemargin}{-.5in}
%\setlength{\evensidemargin}{-.5in}
%\magnification = \magstephalf
%\nopagenumbers
%\parindent=0 pt
\begin{document}

\centerline{\large\textsc{Student Syllabus\hspace{6mm}Spring 2018}}\vspace{1.5mm}
\centerline{\large\hfill\textsc{MAS 3105 -- Applied Linear Algebra}\hfill}\vspace{3mm}
 
\noindent \textbf{INSTRUCTOR:} Chris Stover\\[1.5mm]
\noindent \textbf{EMAIL:} \texttt{cstover@math.fsu.edu}\\[1.5mm]
\noindent \textbf{OFFICE:} MCH 402-F\\[1.5mm]
\noindent \textbf{OFFICE HOURS:} To Be Announced, $\underbrace{\text{or by appointment}}_{\substack{\text{I'm flexible!}\\\text{Email for accommodations!}}}$%\\[-7mm]

\begin{center}
	\line(1,0){200}
\end{center}
%\vspace{-1mm}

\noindent \textbf{MEETING INFO:} Mondays (5:15p to 6:05p), Tuesdays, \& Thursdays (both 5:15p to 6:30p) @ 222 MCH

\noindent \textbf{SECTION NUMBER:} 4%\\

\noindent \textbf{CREDIT HOURS:} 4

\noindent \textbf{COURSE WEB PAGE:} \url{http://www.math.fsu.edu/~cstover/teaching/sp18_mas3105/}%\\

\vspace{-3mm}
\ul{Note:} You are encouraged to bookmark this website and refer to it several times per week: I consider it a living document, and as such, it's liable to change on very short notice. \textit{Failure to refer to this web page, check your email, etc., on a regular basis \textbf{will} put you at a disadvantage! \textbf{Don't miss out on information!}}

\vspace{-6mm}
\begin{center}
	\line(1,0){200}
\end{center}
%\vspace{-1mm}

\noindent \textbf{ELIGIBILITY:} You must have the course prerequisites listed below, and must never have completed with a grade of C- or better a course for which MAS 3105 is a (stated or implied) prerequisite. It is the student's responsibility to check and prove eligibility.

\noindent \textbf{PREREQUISITES:} You must have passed MAC 2312 (Calculus II) with a grade of C- or better. %You must have passed MAC 2312 (Calculus II) with a grade of B- or better \textbf{or} passed MAC 2313 (Calculus III) with a grade of C- or better.

\vspace{-2mm}
\begin{center}
	\line(1,0){200}
\end{center}
%\vspace{-1mm}

\noindent \textbf{REQUIRED TEXT:} \textit{Linear Algebra and Its Applications}, 5th Edition, by Lay. ISBN-10: \texttt{0-321-98238-X}.

\noindent \textbf{COURSE CONTENT:} The course will cover basic topics in linear algebra such as vector spaces and subspaces, linear transformations, Gaussian elimination, matrix algebra, determinants, eigenvalues and eigenvectors, inner products, and applications such as the least squares problems. This material will be drawn from Chapters 1--7 of the text. %Chapters 2, 3, 6, and 5 of the text.

\noindent \textbf{COURSE DESCRIPTION:} This course introduces students to key notions and methods that are fundamental to a variety of problems in data analysis and modeling. Lectures, supplemented by the textbook, homework exercises, and quizzes, will emphasize useful concepts and techniques for simplifying and solving large classes of mathematical problems. This course will also expose students to some of the principles behind the methods.%This course covers various types of first-order differential equations, second- and higher-order linear differential equations, systems of first-order equations, power series solutions, Laplace transforms, and numerical methods. More topics may be covered as time permits.

\noindent \textbf{COURSE OBJECTIVES:} The purpose of this course is to introduce students to the ideas, notions, and applications of linear algebra, particularly the rich interplay between algebra and geometry upon which the topic is founded.\vspace{-3mm}

%In particular, this course will draw contrasts between the perception of the theory of ODE (i.e. that ``all differential equations'' can ``be solved'') and the actuality of it; throughout, this will be used as a guiding principle by which a variety of ODE-related techniques (qualitative, analytic, geometric,...) will be presented. \vspace{-3mm}

The material in this course should be mastered before the student proceeds to courses for which it is a prerequisite.

%\newpage

\noindent \textbf{GRADING:} Your grade in the course will be the weighted average of your performance on: \begin{enumerate*}[label=(\alph*)]\item \ul{three} exams; \item weekly-ish quizzes; and \item a final exam.\end{enumerate*}\vspace{-3mm}

Numerical course grades will be determined according to the formula
$$\frac{3x+y+2z}{6},$$
%$$\frac{4x+2y}{6}\left(=\frac{2x+y}{3}\text{, if you're good with fractions} \right),$$
where $x=(\text{midterm exam score})$, $y=(\text{average of quizzes})$, and $z=(\text{final exam score})$. Letter grades will be determined from numerical scores as follows: \vspace{-3mm}
\begin{center}
	A=90--100;\quad\quad\quad B=80--89;\quad\quad\quad C=70--79;\quad\quad\quad D=60--69;\quad\quad\quad F=0--59.
\end{center}\vspace{-3mm}
Plus or minus grades may be assigned. A grade of ``I'' \ul{will not} be given to avoid a grade of F \textit{or} to give additional study time, and failure to process a course drop will result in a course grade of F.

\noindent \textbf{EXAM SCHEDULE:} Below is the tentative (!!!) exam schedule. These dates (except for the final exam) are subject to change.\vspace{-3mm}

\indent \ul{Test 1:} Thursday, February 1
\\[1.5mm]
\indent \ul{Test 2:} Thursday, March 1
\\[1.5mm]
\indent \ul{Test 3:} Thursday, March 29
\\[3mm]
\indent \ul{Final Exam:}\,\,\, $\underbrace{\text{Wednesday, May 2, 5:30pm to 7:30pm}}_\text{Not tentative and 100\% set in stone!}$

\noindent \textbf{EXAM POLICY:} No makeup tests or quizzes will be given. If a test absence is excused, then the final exam score may be substituted for a missing test grade. If a quiz absence is excused, then the next unit test grade will be used for the missing grade. An unexcused absence from a unit test will be penalized. An unexcused absence from a quiz will result in a grade of zero. 

\vspace{-2mm}
\begin{center}
	\line(1,0){200}
\end{center}
%\vspace{-1.75mm}

\noindent \textbf{UNIVERSITY ATTENDANCE POLICY:} Excused absences include documented illness, deaths in the family and other documented crises, call to active military duty or jury duty, religious holy days, and official University activities. These absences will be accommodated in a way that does not arbitrarily penalize students who have a valid excuse. Consideration will also be given to students whose dependent children experience serious illness.

\noindent \textbf{TUTORING FOR MATH:} Tutoring is available for this course via ACE Tutoring at the Learning Studio in the William Johnston Building.  Appointments may be made, and drop-ins are welcome for one-on-one and group tutoring.  Please contact the ACE Learning Studio at \texttt{tutor@fsu.edu}, 850-645-9151, or find more information at \texttt{http://ace.fsu.edu/tutoring}.

\noindent \textbf{ACADEMIC HONOR POLICY:} The Florida State University Academic Honor Policy outlines the University's expectations for the integrity of students' academic work, the procedures for resolving alleged violations of those expectations, and the rights and responsibilities of students and faculty members throughout the process. \vspace{-3mm}

Students are responsible for reading the Academic Honor Policy and for living up to their pledge to ``...be honest and truthful and ... [to] strive for personal and institutional integrity at Florida State University." (Florida State University Academic Honor Policy, found at \url{http://fda.fsu.edu/Academics/Academic-Honor-Policy}.)

\noindent \textbf{AMERICANS WITH DISABILITIES ACT:} Students with disabilities needing academic accommodation should: \begin{enumerate*}[label=(\arabic*), topsep=1.5mm, itemsep=1.5mm]\item register with and provide documentation to the Student Disability Resource Center; and \item bring a letter to the instructor indicating the need for accommodation and what type.\end{enumerate*}\vspace{-3mm}

Please note: \ul{Instructors are not allowed to provide classroom accommodation to a student until appropriate verification from the Student Disability Resource Center has been provided!}

\noindent This syllabus and other class materials are available in alternative format upon request.

\noindent \textbf{FSU STUDENT DISABILITY RESOURCE CENTER:} For more information about services available to FSU students with disabilities, contact

{\centering
 Student Disability Resource Center
\\
874 Traditions Way
\\
108 Student Services Building, 
Florida State University
\\
Tallahassee, FL 32306-4167
\\
(850) 644-9566 (voice)
\\
(850) 644-8504 (TDD)
\\
\url{sdrc@admin.fsu.edu}
\\
\url{http://www.disabilitycenter.fsu.edu/}
\\
}

\vspace{12pt}

\noindent \textbf{SYLLABUS CHANGE POLICY:} This syllabus is a guide for the course and is subject to change with advance notice.

\newpage

\noindent \textbf{LEARNING OBJECTIVES:} While many of you \textit{aren't} math majors, one thing this course will aspire toward is to: \vspace{-3mm}\begin{center}\fbox{teach students to think like mathematicians (when the time is right)!}\end{center}\vspace{-1.5mm}

\noindent My goal isn't to \textit{turn you into} mathematicians, but rather to supplement your existing skill sets with an added degree of mathematical competency and sophistication to help you think like mathematicians when doing so is appropriate. 

\noindent In order to achieve this, we will:\vspace{-1.5mm}
\begin{enumerate}[label=(Obj \arabic*),leftmargin=0.75in,rightmargin=0.25in,itemsep=1.5mm]
	\item \textit{Use your existing knowledge, skill sets, and proclivities to help you best attain understanding the material of this course.}
	
	As mentioned above, many of you \ul{aren't} (aspiring-)mathematicians, and as such, it would be a shame for the organization of this course to approach its aim at the expense of what you're already capable of. 
	
	For that reason, I will make a concerted effort to translate linear algebra \textit{stuff} into a language that draws from your own knowledge and abilities via carefully-tailored lectures and problem sets.
	
	\item \textit{Embrace the interplay between algebra and geometry.}
	
	Generally speaking, there is a rich interplay between algebra and geometry which exists throughout mathematics: Linear algerba is a topic both founded upon this relationship and enriched by it. 
	
	When you do a computation, the goal will be to understand the geometry underlying it; when you conceptualize geometrically, you should also visualize which computations are associated to it.
	
	In this course, assignments/problem sets/etc. will be crafted which require both computational proficiency and geometric understanding, and these will be graded with the aim of helping you to achieve the perfect balance.
	 
	\item \textit{Reinforce old techniques so that previous roadblocks aren't current obstacles.}
	
	Throughout, students will be expected to strengthen their mathematical (both computational and logical) skills.
	
	Success in this area will be achieved by regularly reviewing ``old'' techniques and ideas from the perspective of current material---including removing the veil from ``new'' material to uncover disguised versions of what you've already seen---and progress will be measured by performance on homeworks, quizzes, and tests. 

	\item \textit{Develop the intuition to understand the subtleties and caveats of a problem at (or near) first glance.} 
	
	Some vector space problems are easy to understand/solve; others aren't. Via careful in-class examination, we'll form and analyze these distinctions, and by way of directed out-of-class assignments/readings, students will develop the ability to perform this differentiation by themselves.
	
	\item \textit{Develop the intuition to know precisely {\normalfont{which}} tools to use for a given problem.}
	
	Mathematicians (including part-time mathematicians, like you) don't want to redo the same problem multiple times because they started on the wrong foot! By semester's end, we will aspire to identify which methods work in solving a given problem and which don't; to help make this more tractable, I will incorporate \textit{why?}/\textit{why not?} discussions into the examples presented so that the lessons can be learned firsthand.
\end{enumerate}

%%\vspace{-7mm}
%\begin{center}
%	\line(1,0){200}
%\end{center}
%%\vspace{-1.75mm}
\end{document}