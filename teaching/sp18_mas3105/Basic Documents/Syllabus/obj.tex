\documentclass[12pt,oneside]{amsart}
\usepackage[margin=0.5in]{geometry}
\usepackage{lmodern}
\usepackage[T1]{fontenc}
\usepackage{amsmath,amssymb,xypic,xspace,graphicx,url,changepage,stoversymb}
\usepackage{nopageno}
\usepackage[inline]{enumitem}
\usepackage{soul}
\parskip=12pt
%\setlength{\textwidth}{7.25in}
%\setlength{\textheight}{9.25in}
%\setlength{\topmargin}{-0.75in}
%\setlength{\oddsidemargin}{-.5in}
%\setlength{\evensidemargin}{-.5in}
%\magnification = \magstephalf
%\nopagenumbers
%\parindent=0 pt
\begin{document}

\noindent \textbf{LEARNING OBJECTIVES:} While many of you \textit{aren't} math majors, one thing this course will aspire toward is to: \vspace{-3mm}\begin{center}\fbox{teach students to think like mathematicians (when the time is right)!}\end{center}\vspace{-1.5mm}

\noindent My goal isn't to \textit{turn you into} mathematicians, but rather to supplement your existing skill sets with an added degree of mathematical competency and sophistication to help you think like mathematicians when doing so is appropriate. 

\noindent In order to achieve this, we will:\vspace{-1.5mm}
\begin{enumerate}[label=(Obj \arabic*),leftmargin=0.75in,rightmargin=0.25in,itemsep=1.5mm]
	\item \textit{Use your existing knowledge, skill sets, and proclivities to help you best attain understanding the material of this course.}
	
	As mentioned above, many of you \ul{aren't} (aspiring-)mathematicians, and as such, it would be a shame for the organization of this course to approach its aim at the expense of what you're already capable of. 
	
	For that reason, I will make a concerted effort to translate linear algebra \textit{stuff} into a language that draws from your own knowledge and abilities via carefully-tailored lectures and problem sets.
	
	\item \textit{Embrace the interplay between algebra and geometry.}
	
	Generally speaking, there is a rich interplay between algebra and geometry which exists throughout mathematics: Linear algerba is a topic both founded upon this relationship and enriched by it. 
	
	When you do a computation, the goal will be to understand the geometry underlying it; when you conceptualize geometrically, you should also visualize which computations are associated to it.
	
	In this course, assignments/problem sets/etc. will be crafted which require both computational proficiency and geometric understanding, and these will be graded with the aim of helping you to achieve the perfect balance.
	 
	\item \textit{Reinforce old techniques so that previous roadblocks aren't current obstacles.}
	
	Throughout, students will be expected to strengthen their mathematical (both computational and logical) skills.
	
	Success in this area will be achieved by regularly reviewing ``old'' techniques and ideas from the perspective of current material---including removing the veil from ``new'' material to uncover disguised versions of what you've already seen---and progress will be measured by performance on homeworks, quizzes, and tests. 

	\item \textit{Develop the intuition to understand the subtleties and caveats of a problem at (or near) first glance.} 
	
	Some vector space problems are easy to understand/solve; others aren't. Via careful in-class examination, we'll form and analyze these distinctions, and by way of directed out-of-class assignments/readings, students will develop the ability to perform this differentiation by themselves.
	
	\item \textit{Develop the intuition to know precisely {\normalfont{which}} tools to use for a given problem.}
	
	Mathematicians (including part-time mathematicians, like you) don't want to redo the same problem multiple times because they started on the wrong foot! By semester's end, we will aspire to identify which methods work in solving a given problem and which don't; to help make this more tractable, I will incorporate \textit{why?}/\textit{why not?} discussions into the examples presented so that the lessons can be learned firsthand.
\end{enumerate}

%%\vspace{-7mm}
%\begin{center}
%	\line(1,0){200}
%\end{center}
%%\vspace{-1.75mm}
\end{document}