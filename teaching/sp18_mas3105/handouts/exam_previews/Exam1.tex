\documentclass[12pt]{article}
\usepackage{changepage,soul,graphicx,graphbox,stoversymb}%,afterpage}
\usepackage[left=0.5in,right=0.5in,bottom=1in,top=0.75in]{geometry}%,showframe=true
\everymath{\displaystyle}

\usepackage{multicol}
\usepackage[many]{tcolorbox}
\usepackage[inline]{enumitem}
\usepackage{amsmath,amsthm}
	\theoremstyle{definition}
	\newtheorem{defn}{Definition}
	
	\newtheoremstyle{underl}{4.5mm}{4.5mm}{}{}{}{\textnormal{.}}{ }{\underline{\thmname{#1}}}
	\theoremstyle{underl}
	\newtheorem*{ex}{Ex}

\thispagestyle{empty}

\newenvironment{mypmatrix}[1]{\renewcommand{\arraystretch}{#1}\begin{pmatrix}}{\end{pmatrix}}
\newenvironment{mybmatrix}[1]{\renewcommand{\arraystretch}{#1}\begin{bmatrix}}{\end{bmatrix}}
\newcommand{\pmat}[1]{\begin{mypmatrix}{1.25}#1\end{mypmatrix}}
\newcommand{\bmat}[1]{\begin{mybmatrix}{1.25}#1\end{mybmatrix}}
\newcommand{\TF}{\textbf{True or False:}~}

\makeatletter
\renewcommand*\env@matrix[1][*\c@MaxMatrixCols c]{%
	\hskip -\arraycolsep
	\let\@ifnextchar\new@ifnextchar
	\array{#1}}
\makeatother

\newcommand{\pmatgrid}[2]{\renewcommand{\arraystretch}{1.25}\begin{pmatrix}[#1] #2\end{pmatrix}}
\newcommand{\bmatgrid}[2]{\renewcommand{\arraystretch}{1.25}\begin{bmatrix}[#1] #2\end{bmatrix}}

\newcommand{\capt}[1]{\begin{adjustwidth}{0.5in}{0.5in}\centering\small\textit{#1}\end{adjustwidth}}
\newcommand{\notebox}[2]
{\begin{tcolorbox}[
		enhanced,
		colback=white,
		colframe=black,
		boxrule=0.5pt,
		arc=0pt,
		top=3mm,
		bottom=3mm, 
		grow to left by=-0.5in,
		grow to right by=-0.5in
	]
	\noindent\textbf{#1}\\
	{#2}
\end{tcolorbox}}
\newcommand{\hintbf}[1]{\textbf{Hint}: #1}

\begin{document}
	\section*{\centering Exam 1 Preview}
	
	\noindent Here's a bit of logistical info about the exam.
	\begin{itemize}[topsep=0.125in,itemsep=0.625mm]
		\item There will be 5--7 questions overall, and some will have multiple parts.
		\item The exam will cover the following textbook sections: 
		\begin{itemize}[topsep=0mm]
			\item \S2.1 (the first stuff we did: matrix multiplication, elements of a matrix, etc.)
			\item \S1.1--1.2 (systems of equations + (R)REF + using (R)REF to solve systems) 
			\item \S1.3 (vector equations, i.e. linear combos + spans)
			\item \S1.4--1.5 ($\sansA\vect{x}=\vect{b}$ and/or $\sansA\vect{x}=\vect{0}$)
			\item \S1.7 (linearly dependent/independent)
		\end{itemize}
		\item You should expect the following question formats:
		\begin{itemize}[topsep=0mm]
			\item computation questions (e.g. using matrices to solve systems from start to finish)
			\item multiple-choice questions
			\item True/False questions (which may or may not require justification).
		\end{itemize}
		The True/False questions will mostly look like those from the textbook (which I include here for those of you without the textbook).
	\end{itemize}
	
	\newpage
	
	\noindent Now, here are some sample questions that you should be able to answer before the exam.
	
	\begin{enumerate}[topsep=0.375in, itemsep=0.375in]
		\item Let $\sansA=\pmat{2 & 0 & -1 & -1 \\ 3 & 1 & 2 & -5 \\ -4 & 0 & 4 & 11}$ be an \ul{augmented matrix}.
		\begin{enumerate}[topsep=3mm]
			\item Put $\sansA$ into Row Echelon Form (REF).	
			\item Put $\sansA$ into Reduced Row Echelon Form (RREF).
			\item Write a system of linear equations associated to $\sansA$ \ul{using $x_1$, $x_2$, etc. as your variables}.
			\item Is the system from part (c) consistent? Why or why not?
			\item Write the solution(s) to the system from part (c) in parametric vector form \ul{or} state that no solutions exist.
			\item \TF $\sansA$ is row equivalent to $\pmat{1 & 0 & 0 & 0 \\ 0 & 1 & 0 & 0 \\ 0 & 0 & 0 & 0}$. \ul{Justify your answer!}
		\end{enumerate}
		
		\item 
		\begin{enumerate}
			\item Draw the span of a single nonzero vector in $\Reals^2$.
			\item Draw the span of two linearly dependent vectors in $\Reals^2$.
			\item Draw the span of two linearly independent vectors in $\Reals^2$.
			\item Draw the span of two linearly independent vectors in $\Reals^3$
			\item Draw a vector in the span of two linearly independent vectors in $\Reals^3$.
			\item Draw a vector linearly independent of two linearly independent vectors in $\Reals^3$.
		\end{enumerate}
		
		\item Let $\sansB=\pmat{1 & 0 \\ 0 & -1 \\ -2 & 0}$ and $\vect{v}=\langle 2,1,-3\rangle$. Provide an example matching each of the following criteria \ul{or} state that no such example exists. \ul{Justify your answer!}
		\begin{enumerate}[topsep=3mm]
			\item Give an example of a matrix $\sansC$ such that the product $\sansB\sansC$ exists but the product $\sansC\sansB$ does not exist.
			\item Give an example of a matrix $\sansD$ such that both products $\sansB\sansD$ and $\sansD\sansB$ exist but $\sansB\sansD\neq\sansD\sansB$.
			\item Give an example of a matrix $\sansE$ such that neither $\sansB\sansE$ nor $\sansE\sansB$ exist.
			\item Give an example of a matrix $\sansF$ such that $\sansF \sansB=\pmat{1 & 0 \\ 0 & 1}$ (``the identity matrix''). Is $\sansB \sansF$ equal to ``the identity matrix'' of the appropriate dimension?
			\item Give an example of a vector $\vect{w}$ such that the dot product $\vect{v}\cdot\vect{w}$ exists and equals 7.
			\item Give an example of a vector $\vect{a}$ such that the vectors $\{\vect{a},\vect{v}\}$ are linearly \ul{independent}.
			\item Give an example of a vector $\vect{b}$ such that the vectors $\{\vect{v},\vect{b}\}$ are linearly \ul{dependent}.
			\item Give an example of a vector $\vect{k}$ such that the vectors $\{\vect{v},\vect{0},\vect{k}\}$ are linearly \ul{dependent}.
		\end{enumerate}
		
		\item Let $\sansA=\pmat{-1 & -2 & -3 \\ 7 & 8 & 9 \\ h & -5 & -6}$, let $\vect{c}_1$, $\vect{c}_2$, and $\vect{c}_3$ denote the columns of $\sansA$, and let $\vect{r}_1$, $\vect{r}_2$, and $\vect{r}_3$ denote the rows of $\sansA$.
		\begin{enumerate}[]
			\item For which value(s) $h$ does $\sansA\vect{x}=\vect{0}$ have only the trivial solution?
			\item For which value(s) $h$ does $\sansA\vect{x}=\vect{0}$ have nontrivial solutions? In this case, express the solutions in terms of one or more free variables.
			\item Let $h$ be as in (a). Give an example of a vector $\vect{u}$ for which the set $\{\vect{c}_1,\vect{c}_2,\vect{c}_3,\vect{u}\}$ is linearly independent \ul{or} state that no such vector exists.
			\item Let $h$ be as in (b). Give an example of a vector $\vect{u}$ for which the set $\{\vect{c}_1,\vect{c}_2,\vect{c}_3,\vect{u}\}$ is linearly independent \ul{or} state that no such vector exists.
			\item Let $h$ be as in (a). Give an example of a vector $\vect{u}$ for which the set $\{\vect{r}_1,\vect{r}_2,\vect{r}_3,\vect{u}\}$ is linearly independent \ul{or} state that no such vector exists.
			\item Let $h$ be as in (b). Give an example of a vector $\vect{u}$ for which the set $\{\vect{r}_1,\vect{r}_2,\vect{r}_3,\vect{u}\}$ is linearly independent \ul{or} state that no such vector exists.
		\end{enumerate}
		
		\item Practice True/False questions by doing problems 1(a)--1(z) from \texttt{Exam1TF.pdf} \ul{except} for parts \mbox{(k)--(n)} and parts \mbox{(x)--(z)}.
	\end{enumerate}
\end{document}