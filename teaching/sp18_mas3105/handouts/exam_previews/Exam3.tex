\documentclass[12pt]{article}
\usepackage{changepage,soul,graphicx,graphbox,stoversymb}%,afterpage}
\usepackage[left=0.5in,right=0.5in,bottom=1in,top=0.75in]{geometry}%,showframe=true
\everymath{\displaystyle}

\usepackage{multicol}
\usepackage[many]{tcolorbox}
\usepackage[inline]{enumitem}
\usepackage{amsmath,amsthm}
	\theoremstyle{definition}
	\newtheorem{defn}{Definition}
	
	\newtheoremstyle{underl}{4.5mm}{4.5mm}{}{}{}{\textnormal{.}}{ }{\underline{\thmname{#1}}}
	\theoremstyle{underl}
	\newtheorem*{ex}{Ex}

\thispagestyle{empty}

\newenvironment{mypmatrix}[1]{\renewcommand{\arraystretch}{#1}\begin{pmatrix}}{\end{pmatrix}}
\newenvironment{mybmatrix}[1]{\renewcommand{\arraystretch}{#1}\begin{bmatrix}}{\end{bmatrix}}
\newcommand{\pmat}[1]{\begin{mypmatrix}{1.25}#1\end{mypmatrix}}
\newcommand{\bmat}[1]{\begin{mybmatrix}{1.25}#1\end{mybmatrix}}
\newcommand{\TF}{\textbf{True or False:}~}

\makeatletter
\renewcommand*\env@matrix[1][*\c@MaxMatrixCols c]{%
	\hskip -\arraycolsep
	\let\@ifnextchar\new@ifnextchar
	\array{#1}}
\makeatother

\newcommand{\pmatgrid}[2]{\renewcommand{\arraystretch}{1.25}\begin{pmatrix}[#1] #2\end{pmatrix}}
\newcommand{\bmatgrid}[2]{\renewcommand{\arraystretch}{1.25}\begin{bmatrix}[#1] #2\end{bmatrix}}
\newcommand{\justify}{\ul{Justify your claim.}}

\newcommand{\capt}[1]{\begin{adjustwidth}{0.5in}{0.5in}\centering\small\textit{#1}\end{adjustwidth}}
\newcommand{\notebox}[2]
{\begin{tcolorbox}[
		enhanced,
		colback=white,
		colframe=black,
		boxrule=0.5pt,
		arc=0pt,
		top=3mm,
		bottom=3mm, 
		grow to left by=-0.5in,
		grow to right by=-0.5in
	]
	\noindent\textbf{#1}\\
	{#2}
\end{tcolorbox}}
\newcommand{\hintbf}[1]{\textbf{Hint}: #1}

\newcounter{Exx}
\newcommand{\exbox}[1]{\stepcounter{Exx}\vspace{3mm}\begin{tcolorbox}[breakable,arc=2pt,boxrule=1pt,right=6mm,top=3mm,bottom=6mm]\noindent\textit{\large\bfseries Example \theExx:\\[3mm]}~#1\end{tcolorbox}}
\begin{document}
	\section*{\centering Exam 3 Preview}
	
	\noindent Here's a bit of logistical info about the exam.
	\begin{itemize}[topsep=0.125in,itemsep=0.625mm]
		\item There will be 5--7 questions overall, and some will have multiple parts.
		\item The exam will cover the following textbook \ul{topics}: 
		\begin{itemize}[topsep=0mm]	
			\item The Inverse Matrix Theorem (\S 2.2, \S 2.9, \S 4.6)
			\item Subspaces (\S 2.8, \S 4.1), Bases (\S 2.8, \S 4.3), and Dimension (\S 2.9, \S 4.5)
			\item Column Space (\S 2.8, \S 4.2), Row Space (\S 4.6), Null Space (\S 2.8, \S 4.2)
			\item Rank (\S 2.9, \S 4.6), Nullity (\S 4.6)
			\item Kernel + range of a linear transform (\S 4.2)
			\item Coordinate systems (\S 4.4), change of coordinates (\S 4.7)
			\item \ul{(Other) Important Theorems:} Rank-Nullity Theorem (\S 4.6), The Spanning Set Theorem (\S 4.3)
		\end{itemize}
		\item You should expect the following question formats:
		\begin{itemize}[topsep=0mm]
			\item computation questions (e.g. using matrices to solve systems from start to finish)
			\item multiple-choice questions
			\item True/False questions (which may or may not require justification).
		\end{itemize}
		The True/False questions will mostly look like those from the textbook (which I include here for those of you without the textbook).
		\item For some of the above topics, your review questions will be from other sources:
		\begin{itemize}[topsep=0mm]
			\item HW3 (\#2--\#5)
			\item Examples 1--5 on the \textit{Column Spaces, Nullity, and all that Jazz} handout
			\item Examples 1--3 on the \textit{Invertible Matrix Theorem II} handout
			\item The (!!!) problems on the \ul{Lecture Notes \& Exercises} tab of the webpage
		\end{itemize}
	\end{itemize}
	
	\vspace{0.225in}
	\begin{center}
		\line(1,0){300}
	\end{center}
	\vspace{0.5in}
	
	\newpage
	
	\noindent Now, here are some sample questions for the \ul{remaining} topics that you should be able to answer before the exam.
	
	\begin{enumerate}[topsep=0.125in, itemsep=0.625in]
		\item Let $\calB=\left\{\vect{b}_1=\pmat{2 \\ 4}, \vect{b}_2=\pmat{4 \\ 2}\right\}$ and $\calC=\left\{\vect{c}_1=\pmat{-11 \\ 2}, \vect{c}_2=\pmat{0 \\ -6}\right\}$ be bases for $\Reals^2$.
		\begin{enumerate}[itemsep=0.25in,topsep=4.5mm]
			\item Find the coordinate change matrices $\sansA_\calB$ and $\sansA_\calC$.
			\item Let $\vect{x}=\pmat{2 \\ 0}$. Compute $\left[\vect{x}\right]_\calB$ and $\left[\vect{x}\right]_\calC$.
			\item Find $\vect{y}$ if $\left[\vect{y}\right]_\calB=\pmat{3 \\ 3}$.
			\item Find $\vect{z}$ if $\left[\vect{z}\right]_\calC=\pmat{1 \\ 1}$.
			\item Find the coordinate-change matrix $\sansA_{\calB\to\calC}$.
			\item Prove that $\sansA_{\calB\to\calC}=\sansA^{-1}_\calC \sansA_\calB$.
			\item Find the coordinate-change matrix $\sansA_{\calC\to\calB}$.
		\end{enumerate}
		
		\item Let $\calB$ and $\calC$ be as above and let $\calD=\{\vect{d}_1,\vect{d}_2\}$, where $\vect{d}_1=\vect{b}_1-\vect{c}_2$ and $\vect{d}_2=\vect{b}_2-\vect{c}_1$.
		\begin{enumerate}[itemsep=0.25in,topsep=4.5mm]
			\item Is $\calD$ a basis for $\Reals^2$? \justify
			\item Draw a diagram which relates $(\Reals^2, \std)$, $(\Reals^2, \calB)$, $(\Reals^2, \calC)$, and $(\Reals^2, \calD)$, where $(\Reals^n,\calX)$ denotes $\Reals^n$ with the coordinate system $\calX$ and where $\std$ denotes the ``standard basis'' $\left\{\pmat{1 & 0}^\sansT, \pmat{0 & 1}^\sansT\right\}$ of $\Reals^2$ 
			\item Does the diagram you drew in part (b) commute? Why or why not?
			
			\begin{flushright}\hintbf{This isn't ``free;'' you have to check stuff here!}\end{flushright}
		\end{enumerate}
		
		\newpage
		
		\item Practice True/False questions by doing Example 1 from the \textit{Invertible Matrix Theorem II} handout; here it is for your convenience!
		\exbox{Mark each of the following questions ``true'' or ``false.'' Throughout, let $\vect{v}_1,\ldots,\vect{v}_p$ be vectors in a nonzero subspace $H$ of $\Reals^n$ and let $S=\{\vect{v}_1,\ldots,\vect{v}_p\}$. \justify
		\begin{enumerate}[label=(\alph*)]
			\item The set of all linear combinations of $\vect{v}_1,\ldots,\vect{v}_p$ is a subspace of $\Reals^n$.
			\item If $\{\vect{v}_1,\ldots,\vect{v}_{p-1}\}$ spans $H$, then $S$ spans $H$.
			\item If $\{\vect{v}_1,\ldots,\vect{v}_{p-1}\}$ is linearly independent, then so is $S$.
			\item If $S$ is linearly independent, then $S$ is a basis for $H$.
			\item If $\vsspan\{S\}=H$, then some subset of $S$ is a basis for $H$.
			\item If $\dim H=p$ and $\vsspan\{S\}=H$, then $S$ cannot be linearly dependent.
			\item A plane in $\Reals^3$ is a two-dimensional subspace.
			\item Row operations on a matrix $\sansA$ can change the linear dependence relations among the rows of $\sansA$.
			\item Row operations on a matrix can change the null space.
			\item The rank of a matrix equals the number of nonzero rows.
			\item If an $m\times n$ matrix $\sansA$ is row equivalent to an echelon matrix $\sansU$ and if $\sansU$ has $k$ nonzero rows, then the dimension of the solution space of $\sansA\vect{x}=\vect{0}$ is $m-k$.
			\item If $\sansB$ is obtained from $\sansA$ by elementary row operations, then $\rank(\sansB)=\rank(\sansA)$.
			\item The nonzero rows of a matrix $\sansA$ form a basis for $\row(\sansA)$.
			\item If matrices $\sansA$ and $\sansB$ have the same $\RREF$, then $\row(\sansA)=\row(\sansB)$.
			\item If $H$ is a subspace of $\Reals^3$, then there is a $3\times 3$ matrix $\sansA$ such that $H=\col(\sansA)$.
			\item If $\sansA$ is $m\times n$ and $\rank(\sansA)=m$, then the linear transformation $\vect{x}\mapsto\sansA\vect{x}$ is one-to-one.
			\item If $\sansA$ is $m\times n$ and the linear transformation $\vect{x}\mapsto\sansA\vect{x}$ is onto, then $\rank(\sansA)=m$.
		\end{enumerate}
		}
		
		\newpage
		
		\item Mark each of the following questions ``true'' or ``false.'' Throughout, let $V$ be a vector space and utilize the notation from question 2 above.
		\begin{enumerate}[label=(\alph*),itemsep=11mm,parsep=3mm]
			\item A change-of-coordinates matrix is always invertible.
			
			\item If $\calB=\{\vect{b}_1,\ldots,\vect{b}_n\}$ and $\calC=\{\vect{c}_1,\ldots,\vect{c}_n\}$ are two bases for a vector space $V$, then the $j$th column of the change-of-coordinates matrix $\sansA_{\calB\to\calC}$ is the coordinate vector $\left[\vect{c}_j\right]_\calB$.
			
			\item If $\vect{x}\in V$ and $\calB$ is a basis of $V$ with $n$ vectors, then the $\calB$-coordinate vector of $\vect{x}$ (aka $\left[\vect{x}\right]_\calB$) is in $(\Reals^n,\std)$.
			
			\item The coordinate change matrix $\sansA_\calB$ satisfies $\left[\vect{x}\right]_\calB=\sansA_\calB\vect{x}$ for $\vect{x}\in V$.
			
			\item If $\calB=\std$ is the standard basis for $\Reals^n$, then the $\calB$-coordinate vector of $\vect{x}\in\Reals^n$ is $\vect{x}$ itself.
			
			\item In some situations, a plane in $\Reals^3$ can be ``isomorphic'' to $\Reals^2$.
			
			\hintbf{Two vector spaces $V$ and $W$ are \textit{isomorphic} if there is a one-to-one linear transformation $\rmTT:V\to W$.}
			
			\item The columns of the matrix $\sansA_{\calB\to\calC}$ are $\calB$-coordinate vectors of the vectors in $\calC$.
			
			\item If $V=\Reals^n$ and $\calC=\std$, then $\sansA_{\calB\to\calC}=\sansA_\calB$.
			
			\item The columns of the matrix $\sansA_{\calB\to\calC}$ are linearly independent.
			
			\item If $V=\Reals^2$, $\calB=\{\vect{b}_1,\vect{b}_2\}$, and $\calC=\{\vect{c}_1,\vect{c}_2\}$, then row reduction of the augmented matrix $\pmatgrid{cc|cc}{\vect{b}_1 & \vect{b}_2 & \vect{c}_1 & \vect{c}_2}$ to $\pmatgrid{c | c}{I_2 & \sansP}$ produces a matrix $\sansP$ which satisfies $\left[\vect{x}\right]_\calB=\sansP\left[\vect{x}\right]_\calC$ for all $\vect{x}\in V$.
		\end{enumerate}
	\end{enumerate}
\end{document}