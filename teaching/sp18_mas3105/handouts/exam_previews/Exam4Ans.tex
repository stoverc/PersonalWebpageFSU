\documentclass[12pt]{article}

\usepackage[margin=0.75in]{geometry}
\usepackage{amsmath,amsfonts,soul}
\usepackage{enumitem}

\newcommand{\vect}[1]{\mathbf{#1}}
\newcommand{\dist}{\operatorname{dist}}
\newcommand{\Reals}{\mathbb{R}}

\begin{document}
	\begin{enumerate}[itemsep=6mm]
		\item (i) is the only one which can happen.
		\item
		\begin{enumerate}[label=(\alph*),listparindent=6mm,itemsep=3mm]
			\item $\vect{x}\cdot\vect{y}=0$
			\item $\|\vect{x}\|=\sqrt{91}$; $\|\vect{y}\|=\sqrt{91}$
			\item $\dist(\vect{x},\vect{y})=\|\vect{x}-\vect{y}\|=\|\langle-3,5,1,5,11,1\rangle\|=\sqrt{182}$
			\item $\operatorname{angle}=\cos^{-1}(0)=\pi/2$
			\item Because $\vect{x}$ and $\vect{y}$ are linearly independent, $W$ has dimension $2$.
			\item Both $\vect{x}$ and $\vect{y}$ live in $\Reals^6$, so the dimension of $W^\perp$ is $6-2=4$.
			\item Write out the equations for $\vect{u}\cdot\vect{x}=0$, $\vect{u}\cdot\vect{y}=0$, $\vect{v}\cdot\vect{x}=0$, and $\vect{v}\cdot\vect{y}=0$ and plug in accordingly to get the right answer.
			
			For example: $\vect{u}\cdot\vect{x}=0\Leftrightarrow u_1+2u_2+3u_3+4u_4+5u_5+6u_6=0$. That's one of the equations; there should be three others.
		\end{enumerate}
		
		\item \begin{enumerate}[label=(\alph*),listparindent=6mm,itemsep=3mm]
			\item This depends on how creative you decided to be, but make sure it's non-diagonal, is $5\times 5$, has real entries, and is symmetric.
			\item Regardless of what you picked for (a), \ul{all} of the eigenvalues should be real. That's a property of symmetric matrices.
			\item Regardless of what you picked for (a), \ul{all} of the eigenspaces should be orthogonal (i.e. have angle $\pi/2$ between them). That's \textit{also} a property of symmetric matrices.
		\end{enumerate}
		
		\item \begin{enumerate}[label=(\alph*),listparindent=6mm,itemsep=3mm]
			\item True
			\item True
			\item False
			\item False
			\item False
			\item True
			\item True
			\item True
			\item True
			\item True
			\item False
			\item True
			\item True
			\item False
			\item True
			\item True
			\item True
			\item True
			\item False
			\item True
			\item True
			\item True
			\item False 
			\item True
			\item False
		\end{enumerate}
	\end{enumerate}
\end{document}