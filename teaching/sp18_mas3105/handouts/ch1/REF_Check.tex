\documentclass[12 pt]{article}

\usepackage{stoversymb}
\usepackage[margin=0.75in, top=0.875in, bottom=0.875in]{geometry}
\usepackage{amsmath,amsfonts,amssymb,url,multicol,graphicx,tikz,soul}
%===makes urls render well===
\usepackage{lmodern}
\usepackage[T1]{fontenc}
%============================
\usepackage{wasysym} % smileys
\usepackage[inline]{enumitem}

%\everymath{\displaystyle}

\makeatletter
\renewcommand*\env@matrix[1][\arraystretch]{%
	\edef\arraystretch{#1}%
	\hskip -\arraycolsep
	\let\@ifnextchar\new@ifnextchar
	\array{*\c@MaxMatrixCols c}}
\makeatother

%\setenumerate{itemsep=0.25in}
%\setlist[enumerate,1]{label=\arabic*., itemsep=0.5in}
%\setlist[enumerate,2]{label={(\alph*)}, itemsep=0.8in}
%\setlist[enumerate,2]{leftmargin=0.5in}
\setlist[enumerate,1]{itemsep=3mm}
\setlist[enumerate,3]{label={\roman*.}}
\setlist[itemize,1]{label=$\circ$, itemsep=0.25in}

\newcommand{\truefalse}[1]{#1\hfill\rule[-1mm]{220pt}{0.75pt}}
\newcommand{\hint}[1]{\vspace{3mm}\textbf{Hint}: #1}
\newcommand{\note}[1]{\ul{Note}: #1}
\newcommand{\infsum}[3]{\sum_{{#1}={#2}}^\infty {#3}}
\newcommand{\comps}[1]{\langle #1_1,#1_2,#1_3\rangle}
\newcommand{\compslong}[3]{\langle #1, #2, #3\rangle}
\newcommand{\ijk}[2]{#1\vect{#2}}
\newcommand{\pmat}[1]{\begin{pmatrix}[1.25]#1\end{pmatrix}}
\newcommand{\bmat}[1]{\begin{bmatrix}[1.25]#1\end{bmatrix}}
\newcommand{\fright}[1]{\vspace{-4mm}\begin{flushright}\parbox{4in}{#1}\end{flushright}\vspace{-6mm}}

\setlength\parskip{0.75mm}	

\graphicspath{ {./../img/} }
\DeclareGraphicsExtensions{.pdf}

\usepackage{etoolbox,refcount}
\usepackage{multicol}
\newcounter{countitems}
\newcounter{nextitemizecount}
\newcommand{\setupcountitems}{%
	\stepcounter{nextitemizecount}%
	\setcounter{countitems}{0}%
	\preto\item{\stepcounter{countitems}}%
}
\makeatletter
\newcommand{\computecountitems}{%
	\edef\@currentlabel{\number\c@countitems}%
	\label{countitems@\number\numexpr\value{nextitemizecount}-1\relax}%
}
\newcommand{\nextitemizecount}{%
	\getrefnumber{countitems@\number\c@nextitemizecount}%
}
\newcommand{\previtemizecount}{%
	\getrefnumber{countitems@\number\numexpr\value{nextitemizecount}-1\relax}%
}
\makeatother    
\newenvironment{AutoMultiColItemize}{%
	\ifnumcomp{\nextitemizecount}{>}{3}{\begin{multicols}{3}}{}%
		\setupcountitems\begin{enumerate}}%
		{\end{enumerate}%
		\unskip\computecountitems\ifnumcomp{\previtemizecount}{>}{3}{\end{multicols}}{}}

\begin{document}
%\begin{flushright}Name: \line(1,0){200}\end{flushright}
\begin{center}
\Large{\textbf{Check Your Understanding---REF \& RREF}}
\end{center}
The purpose of this handout is to ensure that you're okay with the notions of row echelon form (REF) and reduced row echelon form (RREF) introduced in the last handout.

First, recall the definitions:\vspace{3mm}

\noindent\textbf{Definition:} A matrix is in \ul{row echelon form (REF)} if it satisfies the following three properties:
\begin{enumerate}[leftmargin=0.75in,label=(\roman*)]
	\item All nonzero rows are above any rows of all zeros.
	
	\fright{\red{\textbf{Mnemonic:} ``Rows of zeros have to be at the bottom!''}}
	
	\item Each leading (nonzero) entry of a row is in a column to the right of the leading (nonzero) entry of the row above it. 
	
	\vspace{-3mm}\fright{\red{\textbf{Mnemonic:} ``As you go down, leading entries must move to the right!''}}
	
	\item All entries in a column below a leading (nonzero) entry are zeros. 
	
	\fright{\red{\textbf{Mnemonic:} ``Entries \ul{below} leading entries must be zero!''}}
\end{enumerate}\vspace{4.5mm}

\noindent\textbf{Definition:} A matrix is in \ul{reduced row echelon form (RREF)} if it satisfies the following three properties:
\begin{enumerate}[leftmargin=0.75in,label=(\roman*)]
	\item[(i)--(iii)] It is in REF.
	
	\addtocounter{enumi}{3}
	\item The leading (nonzero) entry in each row is 1. 
	
	\fright{\red{\textbf{Mnemonic:} ``Leading entries must equal 1!''}}
	
	\item Each leading 1 is the \ul{only} nonzero entry in its column. 
	
	\fright{\red{\textbf{Mnemonic:} ``Entries \ul{above} leading entries must also be zero!''}}
\end{enumerate}

\begin{center}\vspace{4.5mm}\rule{3in}{0.75pt}\vspace{3mm}\end{center}

The following example (on the next page) is intended to test your understanding of these definitions.

\newpage

\noindent\textbf{Example.} State which of the RREF conditions (i)--(v) each of the following matrices satisfies. 

\note{If a matrix has no rows of all zeros, then it vacuously satisfies (i), and if a leading entry is in the bottom and/or top row, then it vacuously satisfies (iii) and/or (v)!}\vspace{6mm}

\setlength{\columnsep}{1.875in}
\begin{multicols}{3}
\begin{enumerate}[itemsep=1in,label=,leftmargin=-0.125in, rightmargin=-0.125in]
	\item $\sansA=\pmat{0 & 1 & 2 & 3 \\ 1 & 2 & 3 & 4 \\ 0 & 0 & 0 & 1}$
	\item $\sansB=\pmat{2 & 1 & 3 \\ 1 & 1 & 1  \\ 0 & 0 & 0 \\ 0 & 1 & 0 }$
	\item $\sansC=\pmat{2 & 5 \\ 1 & 0}$
	\item $\sansD=\pmat{0 & 1 & 1 & 0 \\ 0 & 1 & 1 & 0 \\ 0 & 0 & 0 & 1}$
	\item $\sansE=\pmat{1 & 1 & 0 & 0 \\ 1 & 1 & 0 & 0 \\ 0 & 0 & 0 & 1}$
	\item $\sansF=\pmat{1 & 1 & 0 & 0 \\ 0 & 1 & 1 & 0 \\ 0 & 0 & 0 & 1}$
	\item $\sansG=\pmat{1 & 0 & 0 & 0 \\ 0 & 1 & 1 & 0 \\ 0 & 0 & 1 & 1}$
	\item $\sansH=\pmat{2 & 0 & 0 \\ 0 & 1 & 2 \\ 0 & 0 & 3}$
	\item $\sansI=\pmat{1 & 0 & 0 & 0 & 0\\ 0 & 1 & 0 & 0 & 0 \\ 0 & 0 & 1 & 0 & 0 \\ 0 & 0 & 0 & 0 & 1 \\ 0 & 0 & 0 & 0 & 1}$
	\item $\sansJ=\pmat{2 & 0 & 0 \\ 0 & 1 & 2 \\ 0 & 0 & 3 \\ 0 & 0 & 0 }$
	\item $\sansK=\pmat{1 & 0 & 0 & 0 \\ 0 & 1 & 0 & 0 \\ 0 & 0 & 0 & 1 \\ 0 & 0 & 0 & 0}$
	\item $\sansL=\pmat{1 & 0 & 0 & 0 \\ 0 & 0 & 1 & 0 \\ 0 & 0 & 0 & 1 \\ 0 & 0 & 0 & 0}$
\end{enumerate}
\end{multicols}
\end{document}