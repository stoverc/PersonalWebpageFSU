\documentclass[12 pt]{article}

\usepackage{stoversymb}
\usepackage[margin=0.75in, top=0.875in, bottom=0.875in]{geometry}
\usepackage{amsmath,amsfonts,amssymb,url,multicol,graphicx,tikz,soul}
%===makes urls render well===
\usepackage{lmodern}
\usepackage[T1]{fontenc}
%============================
\usepackage{wasysym} % smileys
\usepackage[inline]{enumitem}

%\everymath{\displaystyle}

% fancy quotes
\definecolor{quotemark}{gray}{0.7}
\makeatletter
\def\fquote{%
	\@ifnextchar[{\fquote@i}{\fquote@i[]}%]
}%-
\def\fquote@i[#1]{%
	\def\tempa{#1}%
	\@ifnextchar[{\fquote@ii}{\fquote@ii[]}%]
}%
\def\fquote@ii[#1]{%
	\def\tempb{#1}%
	\@ifnextchar[{\fquote@iii}{\fquote@iii[]}%]
}%
\def\fquote@iii[#1]{%
	\def\tempc{#1}%
	\vspace{1mm}%
	\noindent%
	\begin{list}{}{%
			\setlength{\leftmargin}{0.5\textwidth}%
			\setlength{\rightmargin}{3mm}%
		}%
		\item[]%
		\begin{picture}(0,0)%
		\put(-15,-5){\makebox(0,0){\scalebox{5}{\textcolor{quotemark}{``}}\hspace{6mm}}}%
		\end{picture}%
		\begingroup\itshape}%
	%%%%******************************************************
	\def\endfquote{%
		\endgroup\par%
		\makebox[0pt][l]{%
			\hspace{0.5\textwidth}%
			\begin{picture}(0,0)(0,0)%
			\put(15,15){\makebox(0,0){%
					\scalebox{5}{\color{quotemark}''}}}%
			\end{picture}}%
		\ifx\tempa\empty%
		\else%
		\ifx\tempb\empty %
		\hfill\rule{100pt}{0.5pt}\\\mbox{}\hfill\tempa
		\else%
		\ifx\tempc\empty%
		\hfill\rule{100pt}{0.5pt}\\\mbox{}\hfill\tempa,\ {\sc{\tempb}}%
		\else%
		\hfill\rule{100pt}{0.5pt}\\\mbox{}\hfill\tempa,\ {\sc{\tempb}},\ \tempc%
		\fi\fi\fi\par%
		\vspace{0.5em}%
	\end{list}%
}%
\makeatother
%%%%******************************************************

\newenvironment{mypmatrix}[1]{\renewcommand{\arraystretch}{#1}\begin{pmatrix}}{\end{pmatrix}}
\newenvironment{mybmatrix}[1]{\renewcommand{\arraystretch}{#1}\begin{bmatrix}}{\end{bmatrix}}

%\setenumerate{itemsep=0.25in}
%\setlist[enumerate,1]{label=\arabic*., itemsep=0.5in}
%\setlist[enumerate,2]{label={(\alph*)}, itemsep=0.8in}
%\setlist[enumerate,2]{leftmargin=0.5in}
\setlist[enumerate,1]{itemsep=3mm}
\setlist[enumerate,3]{label={\roman*.}}
\setlist[itemize,1]{label=$\circ$, itemsep=0.25in}

\newcommand{\fright}[1]{\vspace{-4mm}\begin{flushright}\parbox{4in}{#1}\end{flushright}\vspace{-6mm}}
\newcommand{\truefalse}[1]{#1\hfill\rule[-1mm]{220pt}{0.75pt}}
\newcommand{\hint}[1]{\vspace{3mm}\textbf{Hint}: #1}
\newcommand{\note}[1]{\ul{Note}: #1}
\newcommand{\infsum}[3]{\sum_{{#1}={#2}}^\infty {#3}}
\newcommand{\comps}[1]{\langle #1_1,#1_2,#1_3\rangle}
\newcommand{\compslong}[3]{\langle #1, #2, #3\rangle}
\newcommand{\ijk}[2]{#1\vect{#2}}
\newcommand{\pmat}[1]{\begin{mypmatrix}{1.25}#1\end{mypmatrix}}
\newcommand{\bmat}[1]{\begin{mybmatrix}{1.25}#1\end{mybmatrix}}

\makeatletter
\renewcommand*\env@matrix[1][*\c@MaxMatrixCols c]{%
	\hskip -\arraycolsep
	\let\@ifnextchar\new@ifnextchar
	\array{#1}}
\makeatother

\newcommand{\pmatgrid}[2]{\renewcommand{\arraystretch}{1.25}\begin{pmatrix}[#1] #2\end{pmatrix}}
\newcommand{\bmatgrid}[2]{\renewcommand{\arraystretch}{1.25}\begin{bmatrix}[#1] #2\end{bmatrix}}

\setlength\parskip{0.75mm}	

\graphicspath{ {./../img/} }
\DeclareGraphicsExtensions{.pdf}


\begin{document}
%\begin{flushright}Name: \line(1,0){200}\end{flushright}
\begin{center}
\Large{\textbf{Check Your Understanding:\\Linear Combinations, Spans, and Solving Linear Systems}}
\end{center}

\begin{fquote}[\small{William Shakespeare}][][]
	\small{What's in a name? that which we call a rose\\
		By any other word would smell as sweet...}
\end{fquote}

As you've probably noticed by now, linear algebra uses a lot of \textit{words}---words and terminology and definitions and notation---and in lots of cases, you can phrase a single concept in lots of different ways.

Such is the case with linear combinations.

Here are a bunch of ways to say the same (very important!!) thing.
\begin{center}
	\hspace{6mm}\line(1,0){200}
\end{center}

\begin{center}
\setlength\parskip{3.75mm}	
\framebox{``$\vect{b}$ can be written as a linear combination of the vectors $\vect{a}_1,\ldots,\vect{a}_n$.''}

means the same as

\framebox{``There exist real numbers $x_1,\ldots,x_n$ such that $\vect{b}=x_1\vect{a}_1+\cdots+x_n\vect{a}_n$.''}

means the same as

\framebox{``$\vect{b}$ is an element of $\vsspan\{\vect{a}_1,\ldots,\vect{a}_n\}$.''}

means the same as

\framebox{``The linear system with augmented matrix $\pmatgrid{c|c|c|c}{\vect{a}_1 & \cdots & \vect{a}_n & \vect{b}}$ has some solution.''}

means the same as

\framebox{``The matrix equation $\sansA\vect{x}=\vect{b}$ has some solution, where $\vect{x}=\pmat{x_1 \\ \vdots \\ x_n}$ for some real numbers $x_1,\ldots,x_n$.''}
\end{center}

\vspace{-1.5mm}
\begin{center}
	\hspace{6mm}\line(1,0){200}
\end{center}

\noindent Don't be surprised if we revisit this list later to make more additions!

\newpage

\noindent\textbf{Sample problems.}\vspace{3mm}

\noindent Throughout, let $\vect{v}=\pmat{1 \\ 2 \\ 3 \\ 4}$, $\vect{u}_1=\pmat{-1 \\ 1 \\ 2 \\ 3}$, $\vect{u}_2=\pmat{0 \\ 2 \\ -2 \\ 1}$, $\vect{u}_3=\pmat{2 \\ 1 \\ 3 \\ 4}$, $\vect{u}_4=\pmat{1 \\ 1 \\ 1 \\ 1}$, and $\vect{u}_5=\pmat{2 \\ 2 \\ 2 \\ 2}$.
\begin{enumerate}
	\item Can $\vect{v}$ be written as a linear combination of the vectors $\vect{u}_1$, $\vect{u}_2$, $\vect{u}_3$, $\vect{u}_4$, $\vect{u}_5$?
	\item Find real numbers $x_1$, $x_2$, $x_3$, $x_4$, $x_5$ so that $\vect{v}=x_1\vect{u}_1+x_2\vect{u}_2+x_3\vect{u}_3+x_4\vect{u}_4+x_5\vect{u}_5$ or state that no such numbers exist.
	\item True or False: $\vect{v}$ is an element of $\vsspan\{\vect{u}_1,\vect{u}_2,\vect{u}_3,\vect{u}_4,\vect{u}_5\}$. How do you know?
	\item For the following linear system, determine whether it is (a) consistent with a unique solution, (b) consistent with non-unique solutions, or (c) inconsistent:
	\[
		\begin{array}{ccccccccccc}
			-x_1 &  &  & + & 2x_3 & + & x_4 & + & 2x_5 & = & 1 \\
			x_1 & + & 2x_2 & + & x_3 & + & x_4 & + & 2x_5 & = & 2 \\
			2x_1 & - & 2x_2 & + & 3x_3 & + & x_4 & + & 2x_5 & = & 3 \\
			3x_1 & + & x_2 & + & 4x_3 & + & x_4 & + & 2x_5 & = & 4
		\end{array}
	\]
	If it is consistent, write its solution(s).
	\item Let $\sansA$ be the $4\times 5$ matrix whose columns are the vectors $\vect{u}_1$, $\vect{u}_2$, $\vect{u}_3$, $\vect{u}_4$, and $\vect{u}_5$. Find a vector $\vect{x}$ such that $\sansA\vect{x}=\vect{v}$.
\end{enumerate}


%\setlength{\columnsep}{1.875in}
%\begin{multicols}{3}
%\begin{enumerate}[itemsep=1in,label=,leftmargin=-0.125in, rightmargin=-0.125in]
%	\item $\sansA=\pmat{0 & 1 & 2 & 3 \\ 1 & 2 & 3 & 4 \\ 0 & 0 & 0 & 1}$
%	\item $\sansA'=\pmatgrid{c|c|c|c}{0 & 1 & 2 & 3 \\ 1 & 2 & 3 & 4 \\ 0 & 0 & 0 & 1}$
%	\item $\sansA=\begin{mypmatrix}{1.25}0 & 1 & 2 & 3 \\ 1 & 2 & 3 & 4 \\ 0 & 0 & 0 & 1\end{mypmatrix}$
%\end{enumerate}
%\end{multicols}
\end{document}