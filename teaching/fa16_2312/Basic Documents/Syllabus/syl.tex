\documentclass[12pt,oneside]{amsart}
\usepackage{lmodern}
\usepackage[T1]{fontenc}
\usepackage{amsmath,amssymb,xypic,xspace,graphicx,url}
\usepackage{nopageno}
\usepackage{enumitem}
\usepackage{soul}
\parskip=12pt
\setlength{\textwidth}{7.25in}
\setlength{\textheight}{9in}
\setlength{\topmargin}{-.5in}
\setlength{\oddsidemargin}{-.5in}
\setlength{\evensidemargin}{0in}
%\magnification = \magstephalf
%\nopagenumbers
%\parindent=0 pt
\begin{document}

%\centerline{\textsc{Student Syllabus} \hfill \textsc{MAC 2312--()} \hfill \textsc{Fall 2016}}

\centerline{\large\textsc{Student Syllabus} \hfill \textsc{MAC 2312---Calculus II} \hfill \textsc{Fall 2016}}\vspace{3mm}

\noindent \textbf{INSTRUCTOR:} Chris Stover%\\

\noindent \textbf{EMAIL:} \texttt{cstover@math.fsu.edu}%\\

\noindent \textbf{OFFICE:} MCH 402-F%\\

\noindent \textbf{OFFICE HOURS:} WF 12:00p--1:30p $\underbrace{\text{or by appointment}}_{\substack{\text{I'm flexible!}\\\text{Email for accommodations!}}}$

%\vspace{-1.5mm}\begin{center}\parbox[c][][c]{0.5\linewidth}{\dotfill}\end{center}\vspace{-3mm}

\noindent \textbf{MEETING INFO:} Mondays---5:15p to 6:05p @ 102 LOV\\
\indent \hspace{35.5mm}Tuesdays \& Thursdays---5:15p to 6:30p @ 102 LOV%\\

\noindent \textbf{COURSE WEBPAGE:} \url{http://www.math.fsu.edu/~cstover/teaching/fa16_2312/}%\\

\noindent \textbf{SECTION NUMBER:} 11%\\

\noindent \textbf{CREDIT HOURS:} 4

%\vspace{-4.5mm}\begin{center}\parbox[c][][c]{0.5\linewidth}{\dotfill}\end{center}\vspace{-3mm}

\noindent \textbf{ELIGIBILITY:} You must have the course prerequisites listed below and must never have completed with a grade of C- or better a course for which MAC 2312 is a (stated or implied) prerequisite.  Students with more than four
hours of prior credit in college calculus are required to reduce the credit for MAC 2312 accordingly. It is the student's responsibility to check and prove eligibility.

\noindent \textbf{PREREQUISITES:} You must have passed MAC 2311 (Calculus I)  with a grade of C- or better or have satisfactorily completed at least four hours of
equivalent calculus courses.

\noindent \textbf{TEXT:} \textit{Calculus: Early Transcendentals}, 7th Edition, by
James Stewart. ISBN-10: \texttt{0538497904}.

\noindent \textbf{COURSE CONTENT:} Chapters 7--11 of the text.

\noindent \textbf{COURSE DESCRIPTION:} This course covers techniques of integration, some applications of integration, some topics in differential equations, some topics in analytic geometry, and the elementary theory of sequences and series. The
material in this course should be mastered before the student proceeds to courses for which it is a prerequisite.

\noindent \textbf{COURSE OBJECTIVES:} The purpose of this course is to introduce students to more advanced topics in the calculus and to some of their applications. The material in this course should be mastered before the student proceeds to
courses for which it is a prerequisite.

\noindent \textbf{GRADING:} There will be four unit tests, regular (lengthy) homework, occasional short quizzes, and a \textit{cumulative} final exam. Numerical course grades will be determined according to the formula\vspace{-1mm}
$$(4x+y+2z)/7,$$ \vspace{-3mm}
where $x=(\text{average of unit tests})$, $y=(\text{average of homework and quizzes})$, and $z=(\text{final exam score})$.

Letter grades will be determined from numerical grades as follows: A=90--100; B=80--89; C=70--79; D=60--69; F=0--59. Plus or minus grades may be assigned. A grade of I will not be given to avoid a grade of F or to give additional study time. Failure to process a course drop will result in a course grade of F.

%\noindent \textbf{GRADING:} There will be four unit tests, occasional short quizzes, and a final exam. Numerical course grades will be determined according to the formula (5U+Q+3F)/9 where U = unit test average, Q = quiz average, and F = final exam. Letter grades will be determined from numerical grades as follows: A: 90-100; B: 80-89; C: 70-79; D: 60-69; F: 0-59. Plus or minus grades may be assigned. A grade of I will not be given to avoid a grade of F or to give additional study time. Failure to process a course drop will result in a course grade of F.

\noindent \textit{Exam Schedule.} Below is the tentative exam schedule. These dates (\textbf{except for the final exam date}) are subject to change.

\indent \textbf{TEST 1:} Thursday, September 15.
\\
\indent \textbf{TEST 2:} Thursday, October 6.
\\
\indent \textbf{TEST 3:} Thursday, November 3.
\\
\indent \textbf{TEST 4:} Thursday, December 1.
\\[3mm]
\indent $\underbrace{\text{\textbf{FINAL:}\;\;\;Thursday, December 15, 5:30pm--7:30pm}}_\text{Not tentative and 100\% set in stone!}$
% Instructor, fill in date and time.

\noindent \textbf{EXAM POLICY:} No makeup tests or quizzes will normally be given. If a test absence is excused, then the final exam score may, at the instructor's discretion, be substituted for the missing test grade. If a quiz absence is excused, then the next unit test grade will be used for the missing grade. An unexcused absence from a unit test will be penalized. An unexcused absence from a quiz will result in a grade of
zero. \ul{Students must bring FSU ID cards to all tests.}

\noindent \textbf{UNIVERSITY ATTENDANCE POLICY:} Excused absences include documented illness, deaths in the family and other documented crises, call to active military duty or jury duty, religious holy days, and official University activities. These absences will be accommodated in a way that does not arbitrarily penalize students who have a valid excuse. Consideration will also be given to students whose dependent children experience serious illness.

\noindent \textbf{TUTORING FOR MATH:} Tutoring is available for this course via ACE Tutoring at the Learning Studio in the William Johnston Building.  Appointments may be made, and drop-ins are welcome for one-on-one and group tutoring.  Please contact the ACE Learning Studio at \texttt{tutor@fsu.edu}, 850-645-9151, or find more information at \texttt{http://ace.fsu.edu/tutoring}.

\noindent \textbf{ACADEMIC HONOR POLICY:} The Florida State University Academic Honor Policy outlines the University's expectations for the integrity of students' academic work, the procedures for resolving alleged violations of those expectations, and the rights and responsibilities of students and faculty members throughout the process. Students are responsible for reading the Academic Honor Policy and for living up to their pledge to ``...be honest and truthful and ... [to] strive for personal and institutional integrity at Florida State University." (Florida State University Academic Honor Policy, found at \texttt{http://fda.fsu.edu/Academics/Academic-Honor-Policy}.)

\noindent \textbf{AMERICANS WITH DISABILITIES ACT:}
Students with disabilities needing academic accommodation should:\vspace{-4.5mm}
\begin{enumerate}[label=(\arabic*)]
\item register with and provide documentation to the Student Disability Resource Center; and
\item bring a letter to the instructor indicating the need for accommodation and what type.
\end{enumerate}\vspace{-4.5mm}
\ul{Please note that instructors are not allowed to provide classroom accommodation to a student until appropriate verification from the Student Disability Resource Center has been provided.}

This syllabus and other class materials are available in alternative format upon request.


For more information about services available to FSU students with disabilities, contact:
\begin{center}
 Student Disability Resource Center
\\
874 Traditions Way
\\
108 Student Services Building, 
Florida State University
\\
Tallahassee, FL 32306-4167
\\
(850) 644-9566 (voice)
\\
(850) 644-8504 (TDD)
\\
\texttt{sdrc@admin.fsu.edu}
\\
\texttt{http://www.disabilitycenter.fsu.edu/}
\end{center}


\noindent \textbf{SYLLABUS CHANGE POLICY:}
Except for changes that substantially affect implementation of the evaluation (grading) statement, this syllabus is a guide for the course and is subject to change with advance notice.
\end{document}




