\documentclass[12 pt]{article}

\usepackage{amsmath}    
\usepackage{amssymb}   
\usepackage{amsthm}
\usepackage{graphicx}

\usepackage{geometry}
 \geometry{
 total={210mm,297mm},
 left=20mm,
 right=20mm,
 top=20mm,
 bottom=20mm,
 }

\usepackage[english]{babel}
\usepackage[utf8]{inputenc}
\usepackage{fancyhdr}
 
\pagestyle{fancy}
\fancyhf{}
\rhead{Name: \hspace{100 pt}}
\lhead{\bf Chapter 11 Intro}
%\rfoot{\thepage}


\begin{document}

A {\bf sequence} is a function $ \left\{ a_{n} \right\}_{n=0}^{\infty}$ whose domain is the set of natural numbers (or some subset) $n=0,1,2,3,\ldots$. We list the elements of a sequence in order as

\[
a_{n}=a_{0},a_{1}, a_{2},a_{3},\ldots
\]
 
Write the first six terms of the following sequences.

\begin{enumerate}

\item ${\displaystyle \left\{ \frac{n}{n+1} \right\}_{n=1}^{\infty}}$

\vspace{40mm}

\item ${\displaystyle \left\{ \frac{ (-1)^n (n+1)}{3^{n}} \right\}_{n=1}^{\infty}}$

\vspace{40mm}

\item $ \left\{ \sqrt{n-3} \right\}_{n=3}^{\infty}$

\vspace{40mm}

\item $ \left\{ \cos \left( \frac{n\pi}{6} \right) \right\}_{n=0}^{\infty}$

\vspace{40mm}

\end{enumerate}

\newpage

Sequences can also be written {\bf recursively}. That means that each term depends on the previous term. Determine the first six terms of each of the recursively defined sequences. 

\begin{enumerate}

\item $a_{1}=4, a_{n}=3+a_{n-1}$. 

\vspace{30mm}

\item $a_{1}=1, a_{2}=-1, a_{n}=a_{n-2}+3\cdot a_{n-1}$.

\vspace{30mm}

\item $a_{1}=3, a_{n}=-2 \cdot a_{n-1}$. 

\vspace{30mm}

\end{enumerate}

Find a formula for the general term $a_{n}$ of the sequence.

\begin{enumerate}

\item $\left\{ 2, 7, 12, 17, \ldots \right\}$

\vspace{35mm}

\item $\left\{ \frac{1}{2}, \frac{1}{4}, \frac{1}{8}, \frac{1}{16}, \ldots \right\}$

\vspace{35mm}

\item $\left\{ \frac{3}{5},-\frac{4}{25}, \frac{5}{125},-\frac{6}{625}, \frac{7}{3125},\ldots \right\}$

\end{enumerate}

\newpage

Determine the limit $\lim_{n \to \infty} a_{n}$ of the sequence, if it exists. 

\begin{enumerate}

\item  ${\displaystyle \left\{ \frac{n}{n+1} \right\}}$

\vspace{40mm}

\item $ \left\{ \sqrt{n-3} \right\}$

\vspace{40mm}

\item $ \left\{ \cos \left( \frac{n\pi}{6} \right) \right\}$

\vspace{40mm}

\item ${\displaystyle \left\{ \frac{1}{2^n} \right\}}$

\end{enumerate}

\newpage

A {\bf series} is the sum of a sequence. A series can be the sum of a certain number of terms or it can be the infinite sum of the entire series. Notationally, we represent a series as

$$ {\displaystyle \sum\limits_{n=1}^{N} a_{n}}$$

The number below $\Sigma$ tells us where to start our addition, the number above tells us where to stop. \\

Determine the following sums:

\begin{enumerate}

\item $ {\displaystyle \sum\limits_{n=1}^{5} \frac{n}{n+1}  } $

\vspace{40mm}

\item $ {\displaystyle \sum\limits_{n=4}^{10} n  } $

\vspace{40mm}

\item $ {\displaystyle \sum\limits_{n=0}^{10} \frac{ (-1)^n (n+1)}{3^{n}} } $

\end{enumerate}

\end{document}