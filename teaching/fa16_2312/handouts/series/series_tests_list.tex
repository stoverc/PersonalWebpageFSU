\documentclass[12pt]{article}

\usepackage{stoversymb,graphicx,soul}
\usepackage[letterpaper, margin=0.5in, top=0.75in, bottom=1in, left=0.5in]{geometry}

%\everymath{\displaystyle}

\usepackage{multicol}
\usepackage[many]{tcolorbox}
\usepackage{tikz}

\title{\vspace{-0.75in}\Large{Tests to determine whether $\ser$ converges or diverges.}\vspace{-0.75in}}
\date{}

\usepackage[inline]{enumitem}
\setlist[enumerate,1]{
	leftmargin=0.5in, 
	rightmargin=0.5in,
	topsep=0mm
}

\setlist[itemize,1]{
	label=$\circ$,
	topsep=0mm
}

%\newcommand{\sectitle}[1]{\noindent\hspace{-0.375in}\vspace{4.5mm}\textbf{\large{#1:}}}
%\newcommand{\subsectitle}[2]{\noindent\ul{#1}:\\[3mm]\indent{#2}\\[6mm]}
\newcommand{\shortlim}{\lim_{n\to\infty}}
\newcommand{\infsum}[3]{\sum_{{#1}={#2}}^\infty {#3}}
\newcommand{\ser}{\infsum{n}{1}{a_n}}
\newcommand{\ans}[1]{\fbox{\textsc{Answer:}} #1}
\newcommand{\ques}[2]{\sectitle{Question {#1}}{#2}}
	
\begin{document}
	\maketitle
	\vspace{-0.25in}
	\begin{center}
	\textit{Throughout, let $f$ be a function satisfying $f(n)=a_n$.}
	\end{center}
	%\vspace{0.0625in}
	\subsection*{The first test you should ever do, always}
	\begin{enumerate}
		\item[] \hspace{-0.25in}The Test for Divergence, aka the``is convergence even possible?!'' test:
		\begin{itemize}
			\item \textit{What you know:} 
			\begin{itemize}
				\item If $\shortlim{a_n}\neq 0$ or $\shortlim{a_n}$ does not exist, your series \textbf{cannot} converge (i.e., it must diverge).
				\item If $\shortlim{a_n}=0$, your series \textbf{may} converge.
			\end{itemize}
		\end{itemize}
	\end{enumerate}
	
	\subsection*{Specialty ``tests'' that work only for very specific cases}
	\begin{enumerate}
		\item Geometric series:
		\begin{itemize}
			\item Can only use when your series looks like $\infsum{n}{0}ar^n$ and/or $\infsum{n}{1}ar^{n-1}$ for constants $a\neq0$  and $r$.
			\item \textit{What you know:} 
			\begin{itemize}
				\item Your series converges \textbf{if and only if} $|r|<1$.
				\item If your series converges, it converges to $\frac{a}{1-r}$
			\end{itemize}
		\end{itemize}
	
		\item $p$-series:
		\begin{itemize}
			\item Can only use when your series looks like $\sum{\frac{1}{n^p}}$ for $p$ a constant.
			\item \textit{What you know:} 
			\begin{itemize}
				\item Your series converges \textbf{if and only if} $p>1$.
			\end{itemize}
		\end{itemize}
	\end{enumerate}
	
	\subsection*{Tests that work for series with \textit{positive terms only}}
	\begin{enumerate}
		\item The integral test:
		\begin{itemize}
			\item Can only use when $f$ is positive, decreasing, and continuous, and when $\int_1^\infty f(x)\,dx$ can be (hopefully-easily) computed.
			\item \textit{What you know:} 
			\begin{itemize}
				\item $\infsum{n}{1}{a_n}$ converges \textbf{if and only if} $\int_1^\infty f(x)\,dx$ converges.
			\end{itemize}
		\end{itemize} 
	
		\item The comparison test:
		\begin{itemize}
			\item Can only use when $\{a_n\}$ ``looks like'' another sequence $\{b_n\}$ where $\{b_n\}$ also has positive terms and where convergence/divergence of $\sum b_n$ is known.
			\item \textbf{Usually}, you want to pick $\{b_n\}$ so that $\sum b_n$ is either a geometric series or a $p$-series.
			\item \textit{What you know:} 
			\begin{itemize}
				\item If $a_n\leq b_n$ for all $n$ and if $\sum b_n$ converges, then $\sum a_n$ \textbf{converges.}
				\item If $a_n\geq b_n$ for all $n$ and if $\sum b_n$ diverges, then $\sum a_n$ \textbf{diverges.}
				\item \textbf{Note:} Knowing that $a_n\geq b_n$ for $\sum b_n$ convergent \textit{or} that $a_n\leq b_n$ for $\sum b_n$ divergent tells you \textbf{nothing}!
			\end{itemize}
		\end{itemize} 
	
		\item The limit comparison test:
		\begin{itemize}
			\item As with the comparison test, you \ul{can} use this when $\{a_n\}$ ``looks like'' another sequence $\{b_n\}$ where $\{b_n\}$ also has positive terms and where convergence/divergence of $\sum b_n$ is known.\vspace{3mm}
			
			\textbf{However:} This test \ul{should} be used either (a) when the comparison test \textbf{doesn't} work, or (b) if you don't like the comparison test (because inequalities are hard). \textit{If the comparison test works, the limit comparison test will work, but not vice versa!}
			\item As with the comparison test, you want to pick $\{b_n\}$ so that $\sum b_n$ is either a geometric series or a $p$-series.
			\item \textit{What you know:} 
			\begin{itemize}
				\item If $\shortlim a_n/b_n=c$ where $0<c<\infty$, then either $\sum a_n$ and $\sum b_n$ \textit{both} converge or they \textit{both} diverge.
				\item If $\shortlim a_n/b_n=c$ where $c=0$ or $c=\infty$, then you know \textbf{nothing!} In particular: You may have ``squinted wrong'' when picking $\{b_n\}$ or your series $\sum a_n$ may genuinely diverge. \textit{You just don't know!}
			\end{itemize}
		\end{itemize} 
	\end{enumerate}

	\subsection*{Tests that work \textit{only} for series with negative terms}
	\begin{enumerate}
		\item[] \hspace{-0.25in}The alternating series test:
		\begin{itemize}
			\item To use, $\sum a_n$ must be an \textit{alternating series}, i.e. $\sum a_n$ must be writable as $\sum (-1)^nb_n$ or $\sum (-1)^{n+1}b_n$ for $\{b_n\}$ a sequence with positive terms.
			\item \textit{What you know:} 
			\begin{itemize}
				\item If $b_n$ is decreasing (i.e. if $b_{n+1}\leq b_n$ for all $n$) if $\shortlim b_n=0$, then $\sum a_n$ \textbf{converges}.
			\end{itemize}
		\end{itemize}
	\end{enumerate}

	\subsection*{Other tests}
	\begin{enumerate}
		\item The ratio test:
		\begin{itemize}
			\item Can use with \textit{any} series.
			\item \textit{What you know:} 
			\begin{itemize}
				\item If $\shortlim|a_{n+1}/a_n|<1$, then $\sum a_n$ \textbf{converges absolutely} (and hence \textbf{converges}).
				\item If $\shortlim|a_{n+1}/a_n|>1$ (or is $\infty$), then $\sum a_n$ \textbf{diverges}.
				\item If $\shortlim|a_{n+1}/a_n|=1$, then $\sum a_n$ the ratio test tells you \textbf{nothing!}
			\end{itemize}
		\end{itemize} 
	
		\item The root test:
		\begin{itemize}
			\item Can use with \textit{any} series.
			\item \textit{What you know:} 
			\begin{itemize}
				\item If $\shortlim\sqrt[n]{|a_n|}<1$, then $\sum a_n$ \textbf{converges absolutely} (and hence \textbf{converges}).
				\item If $\shortlim\sqrt[n]{|a_n|}>1$ (or is $\infty$), then $\sum a_n$ \textbf{diverges}.
				\item If $\shortlim\sqrt[n]{|a_n|}=1$, then $\sum a_n$ the root test tells you \textbf{nothing!}
			\end{itemize}
		\end{itemize} 
	\end{enumerate}
\end{document}