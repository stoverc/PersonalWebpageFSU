\documentclass[10pt]{article}

\usepackage{stoversymb,graphicx,soul}
\usepackage[a4paper, margin=0.5in, top=0.75in, bottom=1in, left=0.75in]{geometry}

\everymath{\displaystyle}

\usepackage{multicol}
\usepackage[many]{tcolorbox}
\usepackage{tikz}

\title{\vspace{-0.75in}\Large{How to determine whether $\ser$ converges or diverges.}\vspace{-0.75in}}
\date{}

\usepackage[inline]{enumitem}
\setlist[enumerate,1]{
	leftmargin=0.75in, 
	rightmargin=0.5in, 
	label=\textbf{Question \arabic*:},
	itemsep=1mm,
	topsep=4.5mm,
	parsep=3mm
}
\setlist[enumerate,2]{
	leftmargin=-0.0625in,
	label=\textsc{Answer:},
	parsep=-1.5mm
}

\setlist[itemize,1]{
	parsep=0in,
	itemsep=2.75mm, 
	topsep=0in,
	leftmargin=-0.5in,
	label=$\circ$
}

\setlist[itemize,2]{
	itemsep=1.5mm,
	leftmargin=0.25in
}
\setlist[itemize,3]{
	leftmargin=0.375in,
	label=$\Longrightarrow$,
	topsep=2.5mm,
	itemsep=2.5mm
}

%\newcommand{\sectitle}[2]{\noindent\textbf{#1:} {#2}\\[3mm]}
%\newcommand{\subsectitle}[2]{\noindent\ul{#1}:\\[3mm]\indent{#2}\\[6mm]}
\newcommand{\shortlim}{\lim_{n\to\infty}}
\newcommand{\infsum}[3]{\sum_{{#1}={#2}}^\infty {#3}}
\newcommand{\ser}{\infsum{n}{1}{a_n}}
\newcommand{\ans}[1]{\fbox{\textsc{Answer:}} #1}
\newcommand{\ques}[2]{\sectitle{Question {#1}}{#2}}
	
\begin{document}
	\maketitle
	\vspace{-0.125in}
	\begin{center}
	\textit{Throughout, let $f$ be a function satisfying $f(n)=a_n$.}
	\end{center}

	\begin{enumerate}
		\item \textit{\ul{Can}} my series converge?
		\begin{enumerate}
			\item Does $\lim_{n\to\infty} a_n$ exist \textit{and} does $\lim_{n\to\infty}a_n=0$?
			\begin{itemize}
					\item If \textit{no}: You're done; $\ser$ \ul{diverges}.
					\item If \textit{yes}: Your series \textit{may} converge. \textbf{Go to Question 2}.
			\end{itemize}
		\end{enumerate}
	
		\item Does my series have negative terms?
			\begin{itemize}
				\item If \textit{no}: You have a positive series. \textbf{Go to Question 3}.
				\item If \textit{yes}: Go to \textbf{Question 5}.
			\end{itemize}
		
		\item Is my series a geometric series or a $p$-series?
			\begin{itemize}
				\item If \textit{yes}: Use the info you know about \textbf{geometric series} and/or \textbf{$\boldsymbol{p}$-series} and you're done.
				\item If \textit{no}: Go to \textbf{Question 4}.
			\end{itemize}
		
		\item If I squint at my series, does it kinda-sorta look like a geometric series or a $p$-series?
%		 {\footnotesize{(For example: $\textstyle\infsum{n}{1}{\frac{n}{2n^3+1}}$ kinda-sorta looks like a $p$-series with $p=2$ while $\textstyle\infsum{n}{1}{\frac{4^{n+1}}{3^n-2}}$ looks a smidgen like a geometric series with $a=16/3$ and $r=4/3$.)}}
			\begin{itemize}
				\item If \textit{yes}, use either \textbf{the comparison test} or \textbf{the limit comparison test}. 
%				\begin{itemize}
%					\item In most cases, either will work.
%					\item The comparison test = easy algebra but perhaps difficult intuition (\textit{WTF do I compare with?!}...see the comparison test for integrals).
%					\item The limit comparison test = easy intuition (just keep the highest power on top and the highest power on bottom) but messy algebra (because dividing + limits).
%				\end{itemize}
				\item If \textit{no}:
				\begin{itemize}
					\item Does my series have factorials and/or $(\text{constant})^n$? 
						\begin{itemize}
							\item \textbf{Use the Ratio Test!}
						\end{itemize}
					\item Does $a_n$ have the form $a_n=(b_n)^n$ ({\footnotesize{a whole function to the $n$th power}})? 
						\begin{itemize}
							\item \textbf{Use the Root Test!}
						\end{itemize}
					\item Does it look like I can find $\textstyle\int_1^\infty f(x)\,dx$? 
						\begin{itemize}
							\item \textbf{(Try to) Use the Integral Test!} ({\footnotesize{$f$ must be \ul{continuous}, \ul{positive}, and \ul{decreasing}!}})
						\end{itemize}
					\item If none of the ratio, root, or integral tests seem appropriate: 
					\begin{itemize}
						\item Ask whatever higher power you believe in for an intervention. ({\footnotesize{If you don't have a higher power, ask a friend to borrow theirs.}})
						\item Try looking at $\textstyle\infsum{n}{1}{|a_n|}$ directly by going back at \textbf{Question 3}.
					\end{itemize}
				\end{itemize}
			\end{itemize}
		\item Is my series alternating? (i.e., is $a_n=(-1)^nb_n$ or $a_n=(-1)^{n+1}b_n$ where $\{b_n\}$ has all positive terms?)
			\begin{itemize}
				\item If \textit{yes}: (Try to) Use the \textbf{Alternating Series Test!} ({\footnotesize{$b_n$ must be decreasing and $\shortlim{b_n}=0$ must hold}})
				\item If \textit{no}:
				\begin{itemize}
					\item Does my series have factorials and/or $(\text{constant})^n$? 
						\begin{itemize}
							\item \textbf{Use the Ratio Test!}
						\end{itemize}
					\item Does $a_n$ have the form $a_n=(b_n)^n$ ({\footnotesize{a whole function to the $n$th power}})? 
						\begin{itemize}
							\item \textbf{Use the Root Test!}
						\end{itemize}
					\item If neither the ratio nor root test seems applicable: 
						\begin{itemize}
							\item See \textbf{Question 4} about borrowing higher powers, etc.
							\item Try looking at $\textstyle\infsum{n}{1}{|a_n|}$ directly by going back at \textbf{Question 3}.
						\end{itemize}
				\end{itemize}
			\end{itemize}
	\end{enumerate}
	
%	\sectitle{Question 1}{\textit{\ul{Can}} my series converge?}
%	\ans{Does $\lim_{n\to\infty} a_n=0$?}
%	\begin{itemize}[itemsep=0in, topsep=0in]
%		\item If \textit{no}, you're done: $\ser$ diverges.
%		\item If \textit{yes}: Your series \textit{may} converge. Go to Question 2.
%	\end{itemize}
%
%	\sectitle{Blah}{Blah}
\end{document}