\documentclass[12pt,oneside]{amsart}
\usepackage{lmodern}
\usepackage[T1]{fontenc}
\usepackage{amsmath,amssymb,xypic,xspace,graphicx,url}
\usepackage{nopageno}
\usepackage[inline]{enumitem}
\usepackage{soul}
\parskip=12pt
\setlength{\textwidth}{7.25in}
\setlength{\textheight}{9.25in}
\setlength{\topmargin}{-.5in}
\setlength{\oddsidemargin}{-.5in}
\setlength{\evensidemargin}{0in}
%\magnification = \magstephalf
%\nopagenumbers
%\parindent=0 pt
\begin{document}

\centerline{\large\textsc{Student Syllabus} \hfill \textsc{MAC 2313---Calculus III} \hfill \textsc{Spring 2017}}\vspace{3mm}

\noindent \textbf{INSTRUCTOR:} Chris Stover%\\

\noindent \textbf{EMAIL:} \texttt{cstover@math.fsu.edu}%\\

\noindent \textbf{OFFICE:} MCH 402-F%\\

\noindent \textbf{OFFICE HOURS:} TBA $\underbrace{\text{or by appointment}}_{\substack{\text{I'm flexible!}\\\text{Email for accommodations!}}}$

\noindent \textbf{MEETING INFO:} Mondays \& Wednesdays---5:15p to 6:05p @ 106 LOV\\
\indent \hspace{35.5mm}Tuesdays \& Thursdays---5:15p to 6:30p @ 106 LOV%\\

\noindent \textbf{COURSE WEB PAGE:} \url{http://www.math.fsu.edu/~cstover/teaching/sp17_2313/}%\\

\vspace{-3mm}
\ul{Note:} I use technology actively, and technology (the course website, Blackboard, email,...) will be utilized regularly throughout this class. \vspace{-3mm}

You are encouraged to bookmark this website and refer to it several times per week: I consider it a living document, and as such, it's liable to change on very short notice.\vspace{-3mm}

\textit{Failure to refer to this web page, check your email, etc., on a regular basis \textbf{will} put you at a disadvantage! \textbf{Don't miss out on information!}}

\noindent \textbf{SECTION NUMBER:} 9%\\

\noindent \textbf{CREDIT HOURS:} 5

\noindent \textbf{ELIGIBILITY:} You must have the course prerequisites listed below and must never have completed with a grade of C- or better a course for which MAC
2313 is a (stated or implied) prerequisite. Students with more than eight hours of prior credit in college calculus are required to reduce the credit for MAC 2313 accordingly. It is the student's responsibility to check and prove eligibility.

\noindent \textbf{PREREQUISITES:} You must have passed MAC 2312 (Calculus II) with a grade of C- or better or have satisfactorily completed at least eight hours of
calculus courses equivalent to MAC 2311 and MAC 2312.

\noindent \textbf{TEXT:} \textit{Calculus: Early Transcendentals}, 7th Edition, by
James Stewart. ISBN-10: \texttt{0538497904}.

\noindent \textbf{COURSE CONTENT:} Chapters 12--16 of the text

\noindent \textbf{COURSE DESCRIPTION:} This course covers functions of several variables and their graphical representations; vectors; partial derivatives and gradients; optimization; multiple integration; polar, spherical, and cylindrical coordinate systems; curves; vector fields; line integrals; flux integrals; divergence theorem and Stokes' theorem.  

\noindent \textbf{COURSE OBJECTIVES:} The purpose of this course is to introduce students to more advanced topics in the calculus and to some of their applications. The material in this course should be mastered before the student proceeds to courses for which it is a prerequisite.

\noindent \textbf{LEARNING OBJECTIVES:} The overarching theme of this section is:\vspace{-3mm}\begin{center}\fbox{Teaching students to think like mathematicians!}\end{center}\vspace{-1.5mm}
%As we'll discover, this is rewarding even though the vast number of you \textit{aren't} aspiring to be mathematicians!\vspace{-3mm}

\noindent In order to achieve this goal, we will aspire to:\vspace{-3mm}
\begin{enumerate}[label=(Obj \arabic*), leftmargin=1in, itemsep=1.5mm]
	\item \textit{Reinforce old techniques so that previous roadblocks aren't current obstacles.} This will be achieved by regularly reviewing those techniques and ideas from the perspective of current material, and success will be measured by performance on homeworks, quizzes, and tests.
	\item \textit{Master ideas (both new and old) so that the {\normalfont{big picture}} becomes clear.} This course is going to require critical thinking in order to appreciate ``the forest'' as an entity independent of ``the trees'', and during, we will encounter a variety of questions which \textit{don't} involve ``plugging stuff in'' or ``just using the formula.''
	\item \textit{Learn to interpret {\normalfont{the new stuff}} as a logical extension of {\normalfont{regular calculus}}}. When lecturing, I'll do my best to make these connections explicit, and problems given for homework will encourage exploration thereof.
	\item \textit{Develop the intuition to know precisely {\normalfont{which}} tools to use for a given problem.} Mathematicians don't want to redo the same problem multiple times because they started on the wrong foot! The goal is to identify which methods work in solving a given problem and which don't; to help make this more tractable, I will incorporate \textit{why?}/\textit{why not?} discussions into the examples presented so that the lessons can be learned firsthand.
\end{enumerate}

\vspace{12pt}

\noindent \textbf{GRADING:} Your grade in the course will the weighted average of your performance on: \begin{enumerate*}[label=(\alph*)]\item \ul{four} unit tests; \item biweekly quizzes; \item regular homework; and \item a final exam\end{enumerate*}

\noindent Numerical course grades will be determined according to the formula
$$\frac{4x+y+2z}{7},$$
where $x=(\text{average of unit tests})$, $y=(\text{average of homework and quizzes})$, and $z=(\text{final exam score})$. Letter grades will be determined from numerical grades as follows: \vspace{-3mm}
\begin{center}
	A=90--100;\quad\quad\quad B=80--89;\quad\quad\quad C=70--79;\quad\quad\quad D=60--69;\quad\quad\quad F=0--59.
\end{center}\vspace{-3mm}
Plus or minus grades \ul{will} be assigned. A grade of I \ul{will not} be given to avoid a grade of F \textit{or} to give additional study time. Failure to process a course drop will result in a course grade of F.

\vspace{12pt}

\noindent \textbf{EXAM SCHEDULE:} Below is the tentative (!!!) exam schedule. These dates (except for the final) are subject to change.\vspace{-3mm}

\indent \ul{Test 1:} Thursday, February 2
\\[1.5mm]
\indent \ul{Test 2:} Thursday, February 23
\\[1.5mm]
\indent \ul{Test 3:} Thursday, March 23
\\[1.5mm]
\indent \ul{Test 4:} Thursday, April 20
\\[3mm]
\indent \ul{Final:}\,\,\, $\underbrace{\text{Wednesday, May 3, 5:30pm to 7:30pm}}_\text{Not tentative and 100\% set in stone!}$

\newpage

\noindent \textbf{EXAM POLICY:} No makeup tests or quizzes will be given. If a test absence is excused, then the final exam score may be substituted for a missing test grade. If a quiz absence is excused, then the next unit test grade will be used for the missing grade. An unexcused absence from a unit test will be penalized. An unexcused absence from a quiz will result in a grade of zero. 

\noindent \textbf{UNIVERSITY ATTENDANCE POLICY:} Excused absences include documented illness, deaths in the family and other documented crises, call to active military duty or jury duty, religious holy days, and official University activities. These absences will be accommodated in a way that does not arbitrarily penalize students who have a valid excuse. Consideration will also be given to students whose dependent children experience serious illness.

\noindent \textbf{TUTORING FOR MATH:} Tutoring is available for this course via ACE Tutoring at the Learning Studio in the William Johnston Building.  Appointments may be made, and drop-ins are welcome for one-on-one and group tutoring.  Please contact the ACE Learning Studio at \texttt{tutor@fsu.edu}, 850-645-9151, or find more information at \texttt{http://ace.fsu.edu/tutoring}.

\noindent \textbf{ACADEMIC HONOR POLICY:} The Florida State University Academic Honor Policy outlines the University's expectations for the integrity of students' academic work, the procedures for resolving alleged violations of those expectations, and the rights and responsibilities of students and faculty members throughout the process. Students are responsible for reading the Academic Honor Policy and for living up to their pledge to ``...be honest and truthful and ... [to] strive for personal and institutional integrity at Florida State University." (Florida State University Academic Honor Policy, found at \texttt{http://fda.fsu.edu/Academics/Academic-Honor-Policy}.)

\noindent \textbf{AMERICANS WITH DISABILITIES ACT:}
Students with disabilities needing academic accommodation should:\vspace{-4.5mm}
\begin{enumerate}[label=(\arabic*), topsep=1.5mm, itemsep=1.5mm, ]
\item register with and provide documentation to the Student Disability Resource Center; and
\item bring a letter to the instructor indicating the need for accommodation and what type.
\end{enumerate}\vspace{-4.5mm}
\ul{Please note that instructors are not allowed to provide classroom accommodation to a student until appropriate verification from the Student Disability Resource Center has been provided.}

\noindent This syllabus and other class materials are available in alternative format upon request.

\noindent For more information about services available to FSU students with disabilities, contact

{\centering
 Student Disability Resource Center
\\
874 Traditions Way
\\
108 Student Services Building, 
Florida State University
\\
Tallahassee, FL 32306-4167
\\
(850) 644-9566 (voice)
\\
(850) 644-8504 (TDD)
\\
\url{sdrc@admin.fsu.edu}
\\
\url{http://www.disabilitycenter.fsu.edu/}
\\
}

\vspace{12pt}

\noindent \textbf{SYLLABUS CHANGE POLICY:}
Except for changes that substantially affect implementation of the evaluation (grading) statement, this syllabus is a guide for the course and is subject to change with advance notice.
\end{document}




