\documentclass[12pt]{article}

\usepackage{stoversymb,graphicx,soul, changepage}
\usepackage[letterpaper, margin=0.5in, top=0.75in, bottom=1in]{geometry}

%\everymath{\displaystyle}

\usepackage{multicol}
\usepackage[many]{tcolorbox}
\usepackage{tikz}

\title{\vspace{-0.75in}\LARGE{Logistical Info about the Final}\vspace{-1in}}
\date{}

\usepackage[inline]{enumitem}
\setlist[enumerate,1]{leftmargin=0.625in, rightmargin=0.5in, label=(\alph*),itemsep=2.25mm,topsep=1.5mm}
\setlist[enumerate,2]{leftmargin=0.25in}
\setlist[itemize]{topsep=-1.5mm}

\renewcommand{\Q}{\vspace{4.5mm}\noindent\textbf{Question}: }
\newcommand{\Ans}{\ul{Answer}: }
\newcommand{\Short}{\ul{Short Answer}: }
\newcommand{\Long}{\vspace{3mm}\ul{Longer Answer}: }
\newcommand{\shortlim}{\lim_{n\to\infty}}
\newcommand{\sectitle}[1]{\vspace{7.5mm}\noindent\textbf{\Large{#1}}\\[3mm]}
\newcommand{\subsectitle}[2]{\vspace{3mm}\noindent\ul{#1}:\\[3mm]\indent{#2}}
\newcommand{\subsecnoindent}[2]{\vspace{3mm}\noindent\ul{#1}:\\[3mm]{#2}}
\newcommand{\comps}[1]{\langle #1_1,#1_2,#1_3\rangle}
\newcommand{\compslong}[3]{\langle #1, #2, #3\rangle}
\newcommand{\pic}[2]{\begin{center}\includegraphics[scale=#1]{#2}\end{center}}
\newcommand{\resultbox}[1]{\begin{center}
		\begin{tcolorbox}[
			enhanced,
			colback=white,
			colframe=black,
			boxrule=0.5pt,
			arc=0pt,
			top=3mm,
			bottom=3mm, 
			width=7in%,
%			grow to left by=0.5in,
			]
			\centering
			#1
		\end{tcolorbox}
\end{center}}
\newcommand{\LH}{L'H\^{o}pital}

\graphicspath{ {./img/} }
\DeclareGraphicsExtensions{.pdf, .png}

\usepackage{caption}
\captionsetup{labelfont=bf, labelformat=simple, justification=centering, labelsep=newline, width=6.5in, textfont={small}}%, textfont={it, footnotesize}}
\captionsetup[figure]{aboveskip=8pt, belowskip=10pt}

\begin{document}
	\setlength\parskip{4.5mm}
	
	\maketitle
	
	\noindent First, a rundown of what kinds of questions will be on your final.
	
	\begin{adjustwidth}{0.25in}{0.25in}
		\noindent \textbf{The new stuff}\\[1.5mm]
		\noindent This will account for approximately 20\%--30\% of the points.
		\begin{itemize}[topsep=1.5mm,parsep=-4.5mm,itemsep=6mm]
			\item one (perhaps multi-part) question on the divergence theorem (\S 16.9);
			\item one (perhaps multi-part) question on Stokes' theorem (\S 16.8);
		\end{itemize}
	
		\noindent\textbf{Test 4 stuff}\\[1.5mm]
		This will account for 20\%--45\% of the points.
		\begin{itemize}[topsep=1.5mm,parsep=-4.5mm,itemsep=6mm]
			\item one question like number 3;
			\item $\geq 1$ line integral (with one requiring Green's theorem);
			\item $\geq 1$ surface integral/flux problem.
		\end{itemize}\vspace{-4.5mm}
		\noindent Note that these can be combined with the new stuff in obvious ways. For example, you may have to find a line integral using the old methods for part (a) of a question, and then find the same line integral in part (b) using Stokes' theorem.\\[-4.5mm]
	
		\noindent\textbf{The old stuff}\\[1.5mm]
		This will account for the remaining points, and not all of the listed concepts will be tested and/or tested equally.
		\begin{itemize}[topsep=1.5mm,parsep=-4.5mm,itemsep=6mm]
			\item using double and triple integrals (which may or may not include polar, cylindrical, or spherical coordinates) to find areas/volumes and volumes/hypervolumes, respectively (e.g. questions 2 and 4 on exam 3);
			\item directional derivatives/gradients (e.g. questions 3b, 3c, and 3d on exam 2);
			\item optimization of 2-variable functions (e.g. question 4 on exam 2);
			\item tangents/normals/curvature/etc. of vector-valued functions (e.g. question 3 on exam 1).
		\end{itemize}
	\end{adjustwidth}
	
	\noindent And now, some random info not covered above.
	
	\begin{adjustwidth}{0.25in}{0.25in}
		\noindent\textbf{Miscellany}
		\begin{itemize}[topsep=1.5mm,parsep=-4.5mm,itemsep=6mm]
			\item there \textbf{won't be} any matching questions (e.g. question 4 on exam 1);
			\item there \textbf{will most likely be} true/false questions; if there are, it will \textbf{only} be about the Chapter 16 stuff;
			\item you \textbf{won't} have to know trig identities;
			\item other than for trig identities, there \textbf{will not} be a formula sheet.
		\end{itemize}
	\end{adjustwidth}
\end{document}