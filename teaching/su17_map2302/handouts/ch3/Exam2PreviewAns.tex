\documentclass[12pt]{article}
\usepackage{changepage,soul,graphicx,graphbox,stoversymb}%,afterpage}
\usepackage[left=0.5in,right=0.5in,bottom=1in,top=0.75in]{geometry}%,showframe=true
\everymath{\displaystyle}

\usepackage{multicol}
\usepackage[many]{tcolorbox}
\usepackage[inline]{enumitem}
\usepackage{amsmath,amsthm}
	\theoremstyle{definition}
	\newtheorem{defn}{Definition}
	
	\newtheoremstyle{underl}{4.5mm}{4.5mm}{}{}{}{\textnormal{.}}{ }{\underline{\thmname{#1}}}
	\theoremstyle{underl}
	\newtheorem*{ex}{Ex}

\thispagestyle{empty}

\newcommand{\capt}[1]{\begin{adjustwidth}{0.5in}{0.5in}\centering\small\textit{#1}\end{adjustwidth}}
\newcommand{\notebox}[2]
{\begin{tcolorbox}[
		enhanced,
		colback=white,
		colframe=black,
		boxrule=0.5pt,
		arc=0pt,
		top=3mm,
		bottom=3mm, 
		grow to left by=-0.5in,
		grow to right by=-0.5in
	]
	\noindent\textbf{#1}\\
	{#2}
\end{tcolorbox}}
\newcommand{\hintbf}[1]{\textbf{Hint}: #1}

\begin{document}
	\section*{\centering Exam 2 Preview Solutions}
	\vspace{0.125in}
	\begin{enumerate}[topsep=0.125in,itemsep=9mm]
%		\item Solve the IVP \[\sin{y}+(x\cos{y}+3y^2)y'=-2x,\quad y(0)=\pi.\]
%		\textbf{Note:} This is the same ODE that was on both Exam 1 \textit{and} Quiz 2; the one on Exam 2 will be different, but at this point, there are still \ul{lots} of people who can't solve this one! If you \textit{can} solve this one, focus on problems 1--15 in \S2.6 as alternatives.
		\item \ul{General Solution}: $x\sin{y}+x^2+y^3=C$
		
		\vspace{3mm}
		\ul{Particular Solution}: $x\sin{y}+x^2+y^3=\pi^3$
		
%		\item The following ODEs are not exact. For each, find an integrating factor which makes it exact.
%		\begin{enumerate}[itemsep=3mm]
%			\item $\left(x y-\cos{y}\right)y'-4=0$
%			\item $\left(4x^2y+2xy+\frac{4}{3}y^3\right)+(x^2+y^2)y'=0$
%		\end{enumerate}
	
		\item \begin{enumerate}[itemsep=4.5mm]
			\item $\frac{N_x-M_y}{M}=\frac{y-0}{-4}=-\frac{y}{4}$ is a function with only $y$'s. Therefore,
			\[m(y)=\exp\left(-\int\frac{y}{4}\,dy\right)=\exp\left(-\frac{y^2}{8}\right)\]
			is your integrating factor. You should verify that multiplying the given ODE by this integrating factor really \textit{does} give an ODE which is exact.
			\item Notice that $\frac{M_y-N_x}{N}=4$, and so this is a function with ``only $x$'s''. Hence,
			\[m(x)=\exp\left(\int 4\,dx\right)=\exp\left(4x\right)\]
			is an integrating factor that works.
		\end{enumerate}
	
%		\item For which of the following initial conditions does the IVP\vspace{-3mm} 
%		\[\left(\tan{y}\right)\dydx+\ln{x}=\ln(\ln(3 x)),\quad y(x_0)=y_0\vspace{-3mm}\]
%		have a unique solution? \textbf{There may be more than one!}
%		\vspace{0mm}
%		\begin{multicols}{3}
%			\begin{enumerate}[itemsep=3mm,leftmargin=0.5in,rightmargin=0.5in,label=\roman*.]
%				\item $y\left(\frac{1}{3}\right)=2$
%				\item $y\left(\frac{1}{6}\right)=2$
%				\item $y\left(\frac{1}{6}\right)=\frac{\pi}{2}$
%				\item $y\left(-\frac{\pi}{2}\right)=\frac{\pi}{4}$
%				\item $y\left(\frac{e}{3}\right)=0$
%				\item $y\left(\frac{e}{3}\right)=\frac{3\pi}{2}$
%				\item $y\left(\frac{\pi}{2}\right)=\frac{3e}{2}$
%				\item $y\left(-\frac{e}{3}\right)=\frac{\pi}{4}$
%				\item None of These
%			\end{enumerate}
%		\end{multicols}
	
		\item \ul{Answer}: (vi) and (vii)
		
		\vspace{3mm}
		\ul{Justification}: Here, 
		\[f(x,y)=\frac{\ln(\ln(3x))-\ln{x}}{\tan{y}}\quad\text{and}\quad\frac{\partial f}{\partial y}(x,y)=-\frac{\sec^2{y}\left(\ln(\ln(3x))-\ln{x}\right)}{\tan^2{y}},\]
		and we observe that anything that breaks $f_y$ also breaks $f$. Note that lots of things could \red{break} $f$:
		\begin{itemize}[itemsep=3mm]
			\item $\tan{y}=0\iff y=n\pi$ for integer $n$ (yields dividing by 0)
			\item $\sin{y}=0\iff y=n\pi$ for integer $n$ (breaks $1/\tan{y}$)
			\item $x\leq 0$ (breaks $\ln{x}$ in numerator)
			\item $3x\leq 0\iff x\leq 0$ (breaks $\ln(3x)$ in numerator)
			\item $\ln(3x)\leq 0\iff 3x\leq 1\iff x\leq 1/3$ (breaks $\ln(\ln(3x))$ in numerator)
		\end{itemize}
		\red{\textbf{Note:} The above list is a list of things which \ul{break} $f$ and/or $f_y$; this is \textit{not} a list of things which make $f$ and/or $f_y$ defined/continuous!}
		
		Combining these things, we see that $x\leq 1/3$ and $y=n\pi$ for integer $n$ both break $f$ (and $f_y$), so we eliminate all answer choices $(x_0,y_0)$ with $x_0\leq 1/3$ and/or $y_0=n\pi$.
		
		The result? (vi) and (vii) are both valid.
		
		\newpage
		
%		\item Solve the following IVPs.
%		\begin{enumerate}
%			\item $y''-4y'+53y=0$, $y(\pi)=\pi$, $y'(\pi)=-\pi$
%			\item $4y''-3y'-7y=0$, $y(0)=0$, $y'(0)=3$
%			\item $y''+2y'+y=0$, $y(1)=2$, $y'(1)=-1$
%		\end{enumerate} 
	
		\item \begin{enumerate}[itemsep=4.5mm]
			\item \ul{General Solution}: $y=e^{2x}\left(c_1\sin(7x)+c_2\cos(7x)\right)$
			
			\vspace{3mm}
			\ul{Particular Solution}: $y=e^{2x}\left(\frac{3}{7}\pi e^{-2\pi}\sin(7x)-\pi e^{-2\pi}\cos(7x)\right)$
			
			\item \ul{General Solution}: $y=c_1e^{7x/4}+c_2e^{-x}$
			
			\vspace{3mm}
			\ul{Particular Solution}: $y=\frac{12}{11}e^{7x/4}-\frac{12}{11}e^{-x}$
			
			\item \ul{General Solution}: $y=c_1e^{-x}+c_2xe^{-x}$
			
			\vspace{3mm}
			\ul{Particular Solution}: $y=e e^{-x}+e x e^{-x}$
		\end{enumerate}
		
%		\item For each of the following functions $y$, determine a second-order linear homoogeneous ODE having $y$ as a general solution. 
%		\begin{enumerate}
%			\item $y=c_1e^{2x}+c_2te^{2x}$
%			\item $y=e^{-3x}\left(c_1\cos{2x}+c_2\sin{2x}\right)$
%			\item $y=c_1e^{-5x}+c_2e^{5x}$
%		\end{enumerate}
	
		\item \begin{enumerate}[itemsep=3mm]
			\item Repeated root $r_1=r_2=2$ gives a characteristic equation $(r-2)(r-2)=0\iff r^2-4r+4=0$. Hence, the ODE is \fbox{$y''-4y'+4y=0$}.
			\item Complex roots $r_1=-3+2i$ and $r_2=-3-2i$: This gives characteristic equation \[(r-r_1)(r-r_2)=0\iff r^2+6r+13=0.\] 
			Thus, we have the corresponding ODE \fbox{$y''+6y'+13y=0$}.
			\item Real non-repeated roots $r_1=-5$ and $r_2=5$ yield a characteristic equation $(r+5)(r-5)=0$, i.e. $r^2-25=0$. This corresponds to the ODE \fbox{$y''-25y=0$}.
		\end{enumerate}
		
%		\item Let $y_1=e^x\sin{3x}$ and $y_2=e^x\cos{3x}$.
%		\begin{enumerate}
%			\item Verify that $y_1$ and $y_2$ are solutions to the ODE $y''-2y'+10y=0$.
%			\item Find the Wronskian $W(y_1,y_2)$.
%			\item Do the functions $y_1$ and $y_2$ constitute a fundamental set of solutions for the ODE $y''-2y'+10y=0$? Why or why not?
%			\item \textbf{True or False:} There exists at least one solution $y_3$ to the ODE $y''-2y'+10y=0$ such that $y_3\neq c_1y_1+c_2y_2$ for any choice of constants $c_1$ and $c_2$. \textbf{Justify your claim!}
%		\end{enumerate}
	
		\item \begin{enumerate}[itemsep=3mm]
			\item You can verify that
			\[y_1'=e^x \sin (3 x)+3 e^x \cos (3 x),\quad y_1''=6 e^x \cos (3 x)-8 e^x \sin (3 x),\]
			\[y_2'=e^x \cos (3 x)-3 e^x \sin (3 x),\quad\text{and}\quad y_2''=-6 e^x \sin (3 x)-8 e^x \cos (3 x).\]
			From here, just plug and chug to show that both $y_1''-2y_1'+10y_1=0$ and $y_2''-2y_2'+10y_2=0$ hold.
			\item $W(y_1,y_2)=-3e^{2x}$.
			\item They \ul{do} form a fundamental system of solutions: By (a), both $y_1$ and $y_2$ solve that ODE, and by (b), $W(y_1,y_2)\neq 0$. Thus, $y_1$ and $y_2$ satisfy both conditions of the definition.
			\item This is \ul{false}: Because $y_1$ and $y_2$ form a fundamental set of solutions, \textit{every} solution $y_3$ of the ODE $y''-2y'+10y=0$ has the form $y_3=c_1y_1+c_2y_2$ for some choice of constants $c_1$ and $c_2$.
		\end{enumerate}
	\end{enumerate}
\end{document}