\documentclass[12pt]{article}
\usepackage{soul,graphicx,stoversymb}%,afterpage}
\usepackage[left=0.5in,right=0.5in,bottom=1in,top=0.75in]{geometry}%,showframe=true
\everymath{\displaystyle}

\usepackage[many]{tcolorbox}

\thispagestyle{empty}

\begin{document}
	\section*{\centering How your proof should look}
	\vspace{3mm}
	{\small{
	I realized that some of you have never had to prove anything before, so I wanted to jump in and give you all some guidance! 
	
	The short story: \textit{To show that two quantities are equal, you start on \textbf{one side} of the equal sign (\textbf{without} messing with the other side) and you do math until the thing you have looks like the \textbf{other side} of the equation!}
			
	Below is an \ul{outline} of how the ``homework'' proof should look: Your job is to \ul{use the justifications provided} (to the right of the blanks) to \ul{fill in the blanks provided}. Once you've filled in all the blanks, what you \textit{should} have is a proof that the integrating factor satisfies the identity claimed in class.
	
	\vspace{2.25mm}
	\noindent\textbf{Note 1:} Throughout, $\exp(x)$ is shorthand for $e^x$ and $I(x)$ is shorthand for $\int p(x)\,dx$. Using this notation,
	$$e^{\,\int p(x)\,dx}=\exp\left(I(x)\right)$$
	is the thing we called $m(x)$ in class and is explicitly highlighted below.

	\vspace{2.25mm}
	\noindent\textbf{Note 2:} ``FTC'' stands for ``Fundamental Theorem of Calculus.''}}
	
	\vspace{12mm}
	\begin{tcolorbox}
		[
		enhanced,
		colback=white,
		colframe=white,
		boxrule=0.5pt,
		arc=0pt,
		top=4.5mm,
		bottom=4.5mm, 
		grow to left by=0.375cm,
		grow to right by=0.375cm,
		enlarge bottom by=0mm,
		center
		]
		\begin{center}
			$\begin{array}{rcll}
				\frac{d}{dx}\left(\underbrace{\exp\left(I(x)\right)}_{m(x)}\,y\right) & = & \underline{\hspace{2in}}\hspace{6mm}&\left(\text{by the product rule}\right)\\[9mm]
					 & = & \underline{\hspace{2in}}\hspace{6mm}&\left(\text{because } \frac{d}{dx}\left(y\right)=\frac{dy}{dx}\right)\\[12mm]
					 & = & \underline{\hspace{2in}}\hspace{6mm}&\left(\text{by the chain rule}\right)\\[12mm]
					 & = & \underbrace{\exp\left(I(x)\right)}_{m(x)}\,\frac{dy}{dx}+\underbrace{\exp\left(I(x)\right)}_{m(x)}p(x)\,y\hspace{6mm}&\left(\text{by FTC}\right)
			\end{array}$
		\end{center}
	\end{tcolorbox}
	\vspace{3mm}
	\vspace*{\fill}%\mbox{}
\end{document}