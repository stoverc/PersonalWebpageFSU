\documentclass[12pt,oneside]{amsart}
\usepackage[margin=0.5in]{geometry}
\usepackage{lmodern}
\usepackage[T1]{fontenc}
\usepackage{amsmath,amssymb,xypic,xspace,graphicx,url,changepage,stoversymb}
\usepackage{nopageno}
\usepackage[inline]{enumitem}
\usepackage{soul}
\parskip=12pt
%\setlength{\textwidth}{7.25in}
%\setlength{\textheight}{9.25in}
%\setlength{\topmargin}{-0.75in}
%\setlength{\oddsidemargin}{-.5in}
%\setlength{\evensidemargin}{-.5in}
%\magnification = \magstephalf
%\nopagenumbers
%\parindent=0 pt
\begin{document}

\centerline{\large\textsc{Student Syllabus\hspace{6mm}Summer 2017}}\vspace{1.5mm}
\centerline{\large\hfill\textsc{MAP 2302 -- Ordinary Differential Equations (ODE)}\hfill}\vspace{3mm}
 
\noindent \textbf{INSTRUCTOR:} Chris Stover\\[1.5mm]
\noindent \textbf{EMAIL:} \texttt{cstover@math.fsu.edu}\\[1.5mm]
\noindent \textbf{OFFICE:} MCH 402-F\\[1.5mm]
\noindent \textbf{OFFICE HOURS:} Tuesdays \& Thursdays -- 12:00p to 1:30p, $\underbrace{\text{or by appointment}}_{\substack{\text{I'm flexible!}\\\text{Email for accommodations!}}}$%\\[-7mm]

\vspace{-1.25mm}
\begin{center}
	\line(1,0){200}
\end{center}
\vspace{-1.75mm}
\noindent Because of its size, this class has a grader assigned. The grader's info is as follows:

%\begin{adjustwidth}{0.25in}{0.25in}
\noindent \textbf{GRADER:} Ryan Vinson\\[1.5mm]
\noindent \textbf{GRADER EMAIL:} \texttt{rvinson@math.fsu.edu}\\[1.5mm]
\noindent \textbf{GRADER OFFICE:} MCH 404/6-A\\[1.5mm]
\noindent \textbf{GRADER OFFICE HOURS:} TBA
%\end{adjustwidth}

\vspace{-4.5mm}
\begin{center}
	\line(1,0){200}
\end{center}
\vspace{-1.75mm}

\noindent \textbf{MEETING INFO:} Mondays, Wednesdays, \& Fridays -- 12:30p to 1:30p @ 101 LOV

\noindent \textbf{SECTION NUMBER:} 1%\\

\noindent \textbf{CREDIT HOURS:} 3

\noindent \textbf{COURSE WEB PAGE:} \url{http://www.math.fsu.edu/~cstover/teaching/su17_map2302/}%\\

\vspace{-3mm}
\ul{Note:} You are encouraged to bookmark this website and refer to it several times per week: I consider it a living document, and as such, it's liable to change on very short notice. \textit{Failure to refer to this web page, check your email, etc., on a regular basis \textbf{will} put you at a disadvantage! \textbf{Don't miss out on information!}}

\vspace{-4.5mm}
\begin{center}
	\line(1,0){200}
\end{center}
\vspace{-1.75mm}

\noindent \textbf{ELIGIBILITY:} You must have the course prerequisites listed below, and must never have completed with a grade of C- or better a course for which MAP 2302 is a (stated or implied) prerequisite (including but not limited to MAP 3305). It is the student's responsibility to check and prove eligibility.

\noindent \textbf{PREREQUISITES:} You must have passed MAC 2312 (Calculus II) with a grade of B- or better \textbf{or} passed MAC 2313 (Calculus III) with a grade of C- or better.

\vspace{-4.5mm}
\begin{center}
	\line(1,0){200}
\end{center}
\vspace{-1.75mm}

\noindent \textbf{REQUIRED TEXT:} \textit{Elementary Differential Equations}, 10th Edition, by Boyce and DiPrima. ISBN-10: \texttt{0-470-45832-1}.

%\noindent \textbf{SUPPLEMENTAL TEXTS:} 
%\vspace{-4.5mm}
%\begin{itemize}[itemsep=1.5mm,leftmargin=9.5mm]
%	\item \textit{Geometrical Methods in the Theory of Ordinary Differential Equations}, 2nd Edition, by Arnol'd. ISBN-10: \texttt{1-461-26994-6}.
%	\item \textit{Differential Equations: A Dynamical Systems Approach: Ordinary Differential Equations}, by Hubbard and West. ISBN-10: \texttt{0-387-97286-2}.
%	\item \textit{Ordinary Differential Equations}, 3rd Edition, by Arnol'd. ISBN-10: \texttt{3-540-54813-0}.
%\end{itemize}

\noindent \textbf{COURSE CONTENT:} Chapters 2, 3, 6, and 5 of the text.

\noindent \textbf{COURSE DESCRIPTION:} This course covers various types of first-order differential equations, second- and higher-order linear differential equations, systems of first-order equations, power series solutions, Laplace transforms, and numerical methods. More topics may be covered as time permits.

\noindent \textbf{COURSE OBJECTIVES:} The purpose of this course is to introduce students to the ideas, notions, and applications of ordinary differential equations (ODE). \vspace{-3mm}

In particular, this course will draw contrasts between the perception of the theory of ODE (i.e. that ``all differential equations'' can ``be solved'') and the actuality of it; throughout, this will be used as a guiding principle by which a variety of ODE-related techniques (qualitative, analytic, geometric,...) will be presented. \vspace{-3mm}

The material in this course should be mastered before the student proceeds to courses for which it is a prerequisite.

\noindent \textbf{LEARNING OBJECTIVES:} The overarching theme of this course is:\vspace{-3mm}\begin{center}\fbox{Teaching students to think like mathematicians!}\end{center}\vspace{-1.5mm}

\noindent In order to achieve this goal, we will aspire to:\vspace{-3mm}
\begin{enumerate}[label=(Obj \arabic*),leftmargin=0.75in,rightmargin=0.25in,itemsep=1.5mm]
	\item \textit{Reinforce old Calculus techniques so that previous roadblocks aren't current obstacles.}
	
	Throughout, students will be expected to strengthen their differentiation and integration skills, as well as their ability to understand/manipulate sequences/series. 
	
	Success in this area will be achieved by regularly reviewing those techniques and ideas from the perspective of current material, and progress will be measured by performance on homeworks, quizzes, and tests. 
%	\item \textit{Master ideas (both new and old) so that the {\normalfont{big picture}} becomes clear.} This course is going to require critical thinking in order to appreciate ``the forest'' as an entity independent of ``the trees'', and during, we will encounter a variety of questions which \textit{don't} involve ``plugging stuff in'' or ``just using the formula.''
%	
%	Unlike traditional calculus, the study of ordinary differential equations is subtle.  
	\item \textit{Learn to interpret {\normalfont{the new stuff}} as a logical extension of {\normalfont{regular calculus}} while recognizing and understanding the (often-subtle) differences between them}. 
	
	Unlike traditional Calculus, the study of ordinary differential equations is subtle. Notions of what it means to ``solve'' such an equation are far less clear-cut than with traditional algebraic equations, and emphasis will be placed on qualitative and numerical techniques in addition to traditional analytic tools.
	\item \textit{Develop the intuition to understand the subtleties and caveats of a problem at (or near) first glance.} 
	
	Some ODE are easy to solve; others aren't. Via careful in-class examination, we'll form and analyze these distinctions, and by way of directed out-of-class assignments/readings, students will develop the ability to perform this differentiation (pun intended) by themselves.
	\item \textit{Develop the intuition to know precisely {\normalfont{which}} tools to use for a given problem.}
	
	Mathematicians don't want to redo the same problem multiple times because they started on the wrong foot! By semester's end, we will aspire to identify which methods work in solving a given problem and which don't; to help make this more tractable, I will incorporate \textit{why?}/\textit{why not?} discussions into the examples presented so that the lessons can be learned firsthand.
\end{enumerate}

\vspace{-7mm}
\begin{center}
	\line(1,0){200}
\end{center}
\vspace{-1.75mm}

\noindent \textbf{GRADING:} Your grade in the course will the weighted average of your performance on: \begin{enumerate*}[label=(\alph*)]\item \ul{four} unit tests; \item weekly-ish quizzes; and \item regular homework.\end{enumerate*}\vspace{-3mm}

Numerical course grades will be determined according to the formula
$$\frac{2x+y}{3},$$
%$$\frac{4x+2y}{6}\left(=\frac{2x+y}{3}\text{, if you're good with fractions} \right),$$
where $x=(\text{average of unit tests})$ and $y=(\text{average of homework and quizzes})$. Letter grades will be determined from numerical scores as follows: \vspace{-3mm}
\begin{center}
	A=90--100;\quad\quad\quad B=80--89;\quad\quad\quad C=70--79;\quad\quad\quad D=60--69;\quad\quad\quad F=0--59.
\end{center}\vspace{-3mm}
Plus or minus grades may be assigned. A grade of ``I'' \ul{will not} be given to avoid a grade of F \textit{or} to give additional study time. Failure to process a course drop will result in a course grade of F.

\noindent \textbf{EXAM SCHEDULE:} Below is the tentative (!!!) exam schedule. These dates (except for the last exam) are subject to change.\vspace{-3mm}

\indent \ul{Test 1:} Friday, June 2
\\[1.5mm]
\indent \ul{Test 2:} Friday, June 23
\\[1.5mm]
\indent \ul{Test 3:} Friday, July 14
\\[3mm]
\indent \ul{Test 4:}\,\,\, $\underbrace{\text{Friday, August 4, 12:30pm to 1:30pm}}_\text{Not tentative and 100\% set in stone!}$

\noindent \textbf{EXAM POLICY:} No makeup tests or quizzes will be given. If a test absence is excused, then the final exam score may be substituted for a missing test grade. If a quiz absence is excused, then the next unit test grade will be used for the missing grade. An unexcused absence from a unit test will be penalized. An unexcused absence from a quiz will result in a grade of zero. 

\vspace{-4.5mm}
\begin{center}
	\line(1,0){200}
\end{center}
\vspace{-1.75mm}

\noindent \textbf{UNIVERSITY ATTENDANCE POLICY:} Excused absences include documented illness, deaths in the family and other documented crises, call to active military duty or jury duty, religious holy days, and official University activities. These absences will be accommodated in a way that does not arbitrarily penalize students who have a valid excuse. Consideration will also be given to students whose dependent children experience serious illness.

\noindent \textbf{TUTORING FOR MATH:} Tutoring is available for this course via ACE Tutoring at the Learning Studio in the William Johnston Building.  Appointments may be made, and drop-ins are welcome for one-on-one and group tutoring.  Please contact the ACE Learning Studio at \texttt{tutor@fsu.edu}, 850-645-9151, or find more information at \texttt{http://ace.fsu.edu/tutoring}.

\noindent \textbf{ACADEMIC HONOR POLICY:} The Florida State University Academic Honor Policy outlines the University's expectations for the integrity of students' academic work, the procedures for resolving alleged violations of those expectations, and the rights and responsibilities of students and faculty members throughout the process. \vspace{-3mm}

Students are responsible for reading the Academic Honor Policy and for living up to their pledge to ``...be honest and truthful and ... [to] strive for personal and institutional integrity at Florida State University." (Florida State University Academic Honor Policy, found at \url{http://fda.fsu.edu/Academics/Academic-Honor-Policy}.)

\noindent \textbf{AMERICANS WITH DISABILITIES ACT:} Students with disabilities needing academic accommodation should: \begin{enumerate*}[label=(\arabic*), topsep=1.5mm, itemsep=1.5mm]\item register with and provide documentation to the Student Disability Resource Center; and \item bring a letter to the instructor indicating the need for accommodation and what type.\end{enumerate*}\vspace{-3mm}

Please note: \ul{Instructors are not allowed to provide classroom accommodation to a student until appropriate verification from the Student Disability Resource Center has been provided!}

\noindent This syllabus and other class materials are available in alternative format upon request.

\newpage

\noindent \textbf{FSU STUDENT DISABILITY RESOURCE CENTER:} For more information about services available to FSU students with disabilities, contact

{\centering
 Student Disability Resource Center
\\
874 Traditions Way
\\
108 Student Services Building, 
Florida State University
\\
Tallahassee, FL 32306-4167
\\
(850) 644-9566 (voice)
\\
(850) 644-8504 (TDD)
\\
\url{sdrc@admin.fsu.edu}
\\
\url{http://www.disabilitycenter.fsu.edu/}
\\
}

\vspace{12pt}

\noindent \textbf{SYLLABUS CHANGE POLICY:}
Except for changes that substantially affect implementation of the evaluation (grading) statement, this syllabus is a guide for the course and is subject to change with advance notice.
\end{document}




