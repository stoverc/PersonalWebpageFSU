\documentclass[12pt]{article}
\usepackage{changepage,soul,graphicx,graphbox,stoversymb}%,afterpage}
\usepackage[left=0.5in,right=0.5in,bottom=1in,top=0.75in]{geometry}%,showframe=true
\everymath{\displaystyle}

\usepackage{multicol}
\usepackage[many]{tcolorbox}
\usepackage[inline]{enumitem}
\usepackage{amsmath,amsthm}
	\theoremstyle{definition}
	\newtheorem{defn}{Definition}
	
	\newtheoremstyle{underl}{4.5mm}{4.5mm}{}{}{}{\textnormal{.}}{ }{\underline{\thmname{#1}}}
	\theoremstyle{underl}
	\newtheorem*{ex}{Ex}

\thispagestyle{empty}

\newcommand{\capt}[1]{\begin{adjustwidth}{0.5in}{0.5in}\centering\small\textit{#1}\end{adjustwidth}}
\newcommand{\notebox}[2]
{\begin{tcolorbox}[
		enhanced,
		colback=white,
		colframe=black,
		boxrule=0.5pt,
		arc=0pt,
		top=3mm,
		bottom=3mm, 
		grow to left by=-0.5in,
		grow to right by=-0.5in
	]
	\noindent\textbf{#1}\\
	{#2}
\end{tcolorbox}}
\newcommand{\hintbf}[1]{\textbf{Hint}: #1}

\begin{document}
	\begin{enumerate}[topsep=0.125in,itemsep=0.625mm]
		\item The IVP always has a solution if $f$ is continuous in a small rectangle containing $x_0$. \red{\ul{True}: $f$ being continuous is enough to conlcude that there's \textbf{a} solution; more is needed for the solution to be unique.}
		\item The IVP always has a \textit{unique} solution if $f$ is continuous in a small rectangle containing $x_0$. \red{\ul{False}: $f$ being continuous is enough to conlcude that there's \textbf{a} solution; more is needed for the solution to be unique.}
		\item The IVP always has a \textit{unique} solution if ${\partial f}/{\partial y}$ is continuous in a small rectangle containing $x_0$. \red{\ul{False}: If $f$ were also continuous, this would be true; however, we can't conclude anything about the continuity of $f$ based ont he continuity of ${\partial f}/{\partial y}$. For example: Imagine that $f$ is a function with \textbf{no $y$'s} which is \textit{also} discontinuous: Then ${\partial f}/{\partial y}=0$ is continuous everywhere but $f$ is discontinuous.}
		\item The IVP always has a solution if $f$ and ${\partial f}/{\partial x}$ are both continuous in a small rectangle containing $x_0$. \red{\ul{True}: $f$ being continuous is enough to conlcude that there's \textbf{a} solution; more is needed for the solution to be unique.}
		\item The IVP always has a \textit{unique} solution if $f$ and ${\partial f}/{\partial x}$ are both continuous in a small rectangle containing $x_0$. \red{\ul{False}: This looks a lot like the existence and uniqueness theorem, except \textbf{that} theorem involves ${\partial f}/{\partial y}$ being continuous, not ${\partial f}/{\partial x}$. A counterexample to this was done in class: We wrote down three solutions to the ODE $y'=y^{1/3}$ in class despite $f=y^{1/3}$ and $\partial f/\partial x=0$ are both continuous everhwhere.}
		\item The IVP may have multiple solutions. \red{\ul{True}: The autonomous ODEs we studied in \S2.5 had multiple solutions (e.g. multiple equilibrium solutions).}
		\item The IVP may have no solution. \red{\ul{True}: This could happen in lots of situations, but one easily-imagined one is that the initial value given can't be plugged in to the solution. For example: If I had $y'=1/x, x>0$ as an ODE, then we can separate and integrate to get $y=\ln(x)+C$ as a general solution. If we turn this into an IVP with the initial condition $y(-5)=2$, however, we'd have something that doesn't exist since we can't plug $x=-5$ into $y=\ln(x)+C$.}
		\item \ul{If} the IVP has a unique solution, the existence and uniqueness theorem tells you that the solution is valid on an $x$-interval containing $x_0$. \red{\ul{True}: This is the only thing that theorem tells you about the solution...}
		\item \ul{If} the IVP has a unique solution, the existence and uniqueness theorem helps you find the $x$-interval containing $x_0$ on which the solution is valid. \red{\ul{False}: ...and this is one of the many things that theorem \textbf{doesn't} tell you! Remember: The existence and uniqueness theorem tells you that if $f$ and $\partial f/\partial y$ are both contiuous in a rectangle containing $(x_0,y_0)$, then the IVP has a unique solution defined on an $x$-interval containing $x_0$; it \textbf{doesn't} tell you \textit{which} $x$-interval!}
		\item If $f(x,y)=0$, then the IVP has a unique solution. \red{\ul{True}: Here, $f=0$ and $\partial f/\partial y=0$ are both continuous everywhere, so you can use the existence and uniqueness theorem. Alternatively, you can also solve this IVP: The general solution would be $y=C$, and using the initial value $y(x_0)=y_0$ tells you that $C=y_0$, and hence that the (unique) particular solution is the constant function $y=y_0$.}
	\end{enumerate}
\end{document}